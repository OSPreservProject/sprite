%-----------------------vdm.tex---------------
\documentstyle[vdm]{article}

\title{Typesetting VDM with \LaTeX}
\author{Mario Wolczko\\
Dept. of Computer Science\\
The University\\
Manchester M13 9PL\\
U.K.\\
\verb;miw@uk.ac.man.cs.ux;, {\tt ...!mcvax!ukc!man.cs.ux!miw}}
\date{November 1986}

\newcommand{\Vdm}{{\tt vdm\/}}

\newenvironment{dangerous}{\endgraf\vspace{5pt}\bgroup\small}%
			  {\endgraf\egroup\vspace{5pt}}

\newlength{\righthalf} \setlength{\righthalf}{0.5\textwidth}
\newlength{\lefthalf}  \setlength{\lefthalf}{0.4\textwidth}
\newenvironment{leftside}{\noindent\hspace{0.1\textwidth}%
			  \parbox[t]{\lefthalf}\bgroup\vspace{10pt}%
			  \noindent\begin{vdm}\leftskip=0pt\VDMindent=0pt}%
			 {\end{vdm}\egroup}
\newenvironment{rightside}{\parbox[t]{\righthalf}\bgroup\begin{verbatim}}%
			  {\egroup}

\renewcommand{\^}[1]{$\langle${\rm #1\/}$\rangle$}

\newcommand{\mmexp}{\^{math-mode-expression}}
\newcommand{\cs}[1]{{\tt \string#1}}

\setlength{\VDMindent}{3\parindent}

\setlength{\preOperationSkip}{0pt}
\setlength{\preFunctionSkip}{0pt}
\setlength{\preTypeSkip}{0pt}
\setlength{\preCompositeSkip}{0pt}
\setlength{\preRecordSkip}{0pt}
\setlength{\preFormulaSkip}{0pt}

\begin{document}
\maketitle

This document describes a new style option, \Vdm, for use with
\LaTeX.  The purpose of \Vdm\ is to make the typesetting of VDM
specifications easy.  Other goals are:
\begin{itemize}
\item	To enable users of \Vdm\ to communicate their specifications
	to others, possibly in a variety of concrete syntaxes, without
	having to change their source files
\item	To help the VDM community standardise on a particular form of VDM.
	This, of course, is in direct contradiction to the first
	aim
\item	To enable a user of \Vdm\ to concentrate on his%
	\footnote{Read `his/her' for every occurence of `his'.}
	specifications, and ignore the detailed layout as much as
	possible.  A side effect of this is that the effort required
	to improve layout is concentrated in one place, within the
	\Vdm\ macros.
\end{itemize}

But enough evangelising.  Let's get to the the real meat.  This
document is broken up into the following sections:
\begin{itemize}
\item	General points about using \Vdm
\item	Typesetting formulas
\item	How to typeset data types
\item	How to typeset functions
\item	How to typeset operations
\item	How to typeset proofs
\item	How to tailor/extend the system for your own
	application.
\end{itemize}
You should definitely read the first two sections---then you'll know
roughly what you're in for, and whether you want to continue.  The
remaining sections can be read as and when you need them.

\begin{dangerous}
In keeping with the best traditions of \TeX\ documentation, paragraphs
that contain material that is not essential for novices, but vital if
you want to parameterise or extend the system, are in smaller type,
like this one.
\end{dangerous}

Just to give a preliminary example, here is some output from \Vdm, and
the corresponding input:

\begin{vdm}
\begin{fn}{dec}{ptrs,om} \\
\signature{
	 \setof{Oop} \x \mapof{Oop}{Object} \to \mapof{Oop}{Object} 
}
\If ptrs = \emptyset
\Then om
\Else 	\Let gone = \set{p \in ptrs | RC(om(p)) = 1} \In
	\Let om' = gone \dsub om \In
	\Let om'' = om' \owr
		\map{p \mapsto \chg{om'(p)}{RC}{RC\minus 1}\T3
			| p \in ptrs \diff gone} \In
	dec(\Union\set{\elems{BODY(om(p))} | p \in gone}, om'')
\Fi
\end{fn}

\begin{op}[DESTROYPTR]
\args{ Obj, Ptr : Oop }
\ext{ \Wr OM : \mapof{Oop}{Basic_Object} }
\pre{ ptr \in  \elems{BODY(om(obj))} }
\post{ om = ~{om} \owr \map{ obj \mapsto 
		\chg{om(obj)}{BODY}{BODY \diff \set{ptr}}}}
\end{op}
\end{vdm}

% this is verbatim input
\begin{verbatim}
\begin{vdm}
\begin{fn}{dec}{ptrs,om} \\
\signature{
      \setof{Oop} \x \mapof{Oop}{Object} \to \mapof{Oop}{Object} 
}
\If ptrs = \emptyset
\Then om
\Else   \Let gone = \set{p \in ptrs | RC(om(p)) = 1} \In
        \Let om' = gone \dsub om \In
        \Let om'' = om' \owr
                \map{p \mapsto \chg{om'(p)}{RC}{RC\minus 1}
                        | p \in ptrs \diff gone} \In
        dec(\Union\set{\elems{BODY(om(p))} | p \in gone}, om'')
\Fi
\end{fn}

\begin{op}[DESTROYPTR]
\args{ Obj, Ptr : Oop }
\ext{ \Wr OM : \mapof{Oop}{Basic_Object} }
\pre{ ptr \in  \elems{BODY(om(obj))} }
\post{ om = ~{om} \owr \map{ obj \mapsto 
                \chg{om(obj)}{BODY}{BODY \diff \set{ptr}}}}
\end{op}
\end{vdm}

\end{verbatim}

\noindent
Impressed, huh?  Read on...


\section*{Using \Vdm---General Points}

To get at \Vdm, include {\tt vdm\/} as a document style option, e.g.:
\begin{verbatim}
        \documentstyle[12pt,vdm]{report}
\end{verbatim}
\begin{dangerous}
To the best of my knowledge, the use of \Vdm\ does not conflict with
any of the other document styles, except when something has been
redefined.   An attempt will be made to document all such redefinitions.
\end{dangerous}
Once \Vdm\ has been included, you can then use the {\tt vdm\/}
environment.  For example,
\begin{verbatim}
         \begin{vdm}
            ....
         \end{vdm}
\end{verbatim}
All specification material should be placed within the {\tt vdm\/}
environment.  The use of \Vdm\ only affects text within the {\tt vdm\/}
environment, except for the following global changes (which are only
relevant when in math or display math mode):
\begin{enumerate}
\item	The mathcodes of a\dots z and A\dots Z have been changed.  In
	plain English, this means that when you type letters in math
	mode the inter-letter spacing is different than it would bed
	had you not included \Vdm\ as an option.  This is because
	\LaTeX\ math mode is usually tuned for single letter
	identifiers, as used by mathematicians for millenia.  However,
	you and I both know that most meaningful identifiers have more
	than one letter in them, so \Vdm\ provides better spacing for
	them.  As an example, if you type \verb;$identifier$;, \LaTeX\
	would normally print {\defaultMathcodes$identifier$}, whereas
	the use of \Vdm\ will yield $identifier$.
	\begin{dangerous}\indent If you really want to use the
	`normal' inter-letter spacing, say \cs\defaultMathcodes.
	\end{dangerous}
\item	Underscore gives you an underscore, and not a subscript.  If
	you want a subscript use \verb;@;, e.g.,~$x@0$ is typed
	\verb;x@0;, or use \TeX's \cs\sb\ macro.  An @ is still an
	@ when not in math mode.  Occasionally you may find that an @
	in math mode {\em doesn't\/} give you a subscript.  Should
	this happen, you are advised to use \TeX's \cs\sb macro,
	e.g.,~\verb;$x\sb0$;. 
	\begin{dangerous}
	If you don't use
	underscores much, and you want to use \verb;_; for subscripts,
	you can say \cs\underscoreon\ (and \cs\underscoreoff\ to
	make it revert to its usual meaning in \Vdm).
	\end{dangerous}

\item	\verb;-; typesets a hyphen, and not a minus sign.  VDM specifications
	usually contain a lot more
	\begin{vdm}$long-identifiers$\end{vdm} than subtractions, so
	on the whole this alteration should save effort.  If you really want
	to do a subtraction, use \cs\minus.

\item	\verb;|; gives you a \begin{vdm}$|$\end{vdm}, and not a $\vert$.
	Do you see the difference?  No?  The former goes between things,
	e.g., \begin{vdm}$\set{x|p(x)}$\end{vdm}, while the latter is
	a delimiter, e.g.,~$|x|$.
	Most people use the former more than the latter, so again this
	seems reasonable.  If you really want a $|$ (the second kind),
	say \cs\vert. 

\item	In \TeX\ and \LaTeX\ \verb;~; has always been a tie.  Well
	in \Vdm\ it isn't.  \verb;~x; will give you a
	\begin{vdm}$~x$\end{vdm}.  For long identifiers, such as
	\begin{vdm}$~{long}$\end{vdm}, say
	\verb;~{long};  {\em Note that this only applies in math
	mode; elsewhere a \verb;~; is still a tie.} 
\item	In math mode, the double quote character \verb;"; is actually
	a macro.  Placing text between pairs of double quotes causes
	that text to be set in the normal text font.  For example,
	\verb;$x="a variable"$; gives you $x="a variable"$.  
	\begin{dangerous}
	If you want to change the font used for text placed between
	quotes, redefine the command \cs\mathTextFont.  By default
	it is defined to be \cs\rm.
	\end{dangerous}
\item	The following macros have been altered in a non-trivial way:
	\cs\forall, \cs\exists.
\end{enumerate}

\begin{dangerous}
When you typeset some VDM within the {\tt vdm\/} environment, by
default it is set in from the left margin by an amount equal to
\cs\parindent, the indentation at the beginning of each paragraph.
If you want to change this, change the value of \cs\VDMindent, e.g.:
\begin{verbatim}
        \setlength{\VDMindent}{0cm}
\end{verbatim}
will make your specs come out flush left.  This document has been
typeset with \cs\VDMindent\ equal to $3\times \cs\parindent$.
\end{dangerous}

\begin{dangerous}
Similarly, you can have a particular line spacing in force within the
{\tt vdm} environment.
%  However, you have to tread carefully here.
%\TeX\ {\em always\/} enforces a single line spacing within a
%particular paragraph, and the value used is the value current at the
%{\em end\/} of the paragraph.  So to make the material within a {\tt
%vdm} environment have a different spacing from that surrounding it,
%you must always put it in a separate paragraph, by placing a blank
%line before and after it, otherwise strange things will happen.  
The spacing within a {\tt vdm} environment is dictated by the
\cs\VDMbaselineskip\ command.  Note that this is {\em not\/} a
length, but a command.  By default it expands to \cs\baselineskip\
so that the line spacing is that of the surrounding text, whatever 
size that may be.  To make it smaller, you may want to say
\begin{verbatim}
    \renewcommand{\VDMbaselineskip}{0.8\baselineskip}
\end{verbatim}
for example.
\end{dangerous}


\section*{Typesetting formulas}

Most of the text you enter within {\tt vdm\/} environments will be in
\TeX's math mode, but VDM does its best to conceal this fact from you,
so that you should rarely, if ever, have to type a dollar sign.
However, several new features have been provided for the typesetting
of logical formulas.  Firstly, operators with sensible names have been
provided: use \cs\Iff, \cs\Implies, \cs\Or, \cs\And\
and \cs\Not\ for the operators~\begin{vdm}%
$\Iff,\Implies,\Or,\And,\Not$\end{vdm}.
(To retain compatibility with a previous version, \cs\iff,
\cs\implies, \cs\and\ and \cs\neg\ are still provided, but
\cs\or\ is not.)

A major change has come in the area of quantified expressions.  They
have very well-defined forms, so the \LaTeX\ sequences \cs\forall\
and \cs\exists\ have been re-defined to take arguments.  For
example, to get
\begin{vdm}
\begin{formula}
\exists{x \in S}{p(x)}
\end{formula}
\end{vdm}
\noindent type
\begin{verbatim}
        \exists{x \in S}{p(x)}
\end{verbatim}
Note the separating dot that was put in auto\-matically.  If you want
one of these dots by itself, you can have one by saying
\cs\suchthat.

In addition, a new quantifier, \cs\unique, has been added:

\begin{leftside}
\begin{formula}
\unique{x \in S}{p(x)}
\end{formula}
\end{leftside}\begin{rightside}
\unique{x \in S}{p(x)}
\end{verbatim}\end{rightside}

\begin{dangerous}
If you want to use the old versions of \cs\forall\ and
\cs\exists\ they are available under the pseudonyms of
\cs\Forall\ and \cs\Exists.
\end{dangerous}

If you find that the body of the quantified expression is too long to
fit comfortably on a line, there are *-forms of the above commands
that place the body of the quantified expression on a new line,
slightly indented.  For example,

\begin{vdm}
\begin{formula}
\exists*{x \in S}{
	p(x) \And q(x) \Or \Not p(x) \Implies r(x) \And S(x)}
\end{formula}
\end{vdm}

\noindent can be obtained with
\begin{verbatim}
      \exists*{x \in S}{p(x) \And q(x) \Or \Not p(x)
                        \Implies r(x) \And S(x)}
\end{verbatim}

If you need ``Strachey'' brackets, e.g., $M\term{e}$, place the
material to appear within the brackets within \verb;\term{ ... };,
thus: \verb;$M\term{e}$;.

A special control sequence, \cs\const, is available for constants.
To get, for example, $\const{Yes}|\const{No}$, type
\verb;\const{Yes}|\const{No};.
\begin{dangerous}
If you don't like the font that constants are set in, you can change
them by redefining the command \cs\constantFont.  By
default it expands to \cs\sc.
\end{dangerous}

\subsection*{The {\tt formula} Environment}

Occasionally you may want a formula on its own, between paragraphs of
text, say.  Normally, the provided environments and commands suffice,
but sometimes they don't.  If you need an odd equation to stand on its
own, use the {\tt formula} environment:
\begin{verbatim}
       \begin{formula}
       x = 10
       \Or  \forall{i \in \Nat}{i \ne 10 \Implies i \ne x}
       \end{formula}
\end{verbatim}
\sloppy
The {\tt formula} environment is similar to displayed math mode,
except: formulas are indented by \cs\VDMindent, not
\cs\mathindent, and line breaks can be made using \cs\\.
Also, within the {\tt formula} environment everything appears flush
left, as opposed to being centred. 


\subsection*{Constructions}

A particularly nice feature of \Vdm\ is that you can typeset multi-line
constructions such as those in the earlier example without having to
worry about, say, lining up ``thens'' and ``elses'' with ``ifs''.
In the following definitions, whenever you see the term \mmexp, you
should type an expression as if in math mode, but you needn't put
dollar signs in.  All of the constructions described below can be used
where a \mmexp\ is required.  Each construction is shown by example;
the output on the left results from the input on the right.
Also note that each macro name begins with an upper-case letter.
\TeX\ and \LaTeX\ frequently use the lower-case variants for
completely unrelated things.  Naturally, chaos will ensue if you mix
the names up.

Typesetting an \kw{if} is done using \cs\If\ \mmexp \cs\Then\ \mmexp
\cs\Else\ \mmexp \cs\Fi.

\begin{leftside}
  \begin{formula}
    \If x\in S
    \Then S \diff x
    \Else \emptyset
    \Fi
  \end{formula}
\end{leftside}%
\begin{rightside}
\If x\in S
\Then S \diff x
\Else \emptyset
\Fi
\end{verbatim}
\end{rightside}

If you nest \cs\If{}s then you must enclose inner \cs\If{}s within
braces:

\begin{leftside}
  \begin{formula}
        \If \ldots
        \Then{
                \If \ldots
                \Then \ldots
                \Else \ldots
                \Fi
        }\Else
        \Fi
  \end{formula}
\end{leftside}\begin{rightside}
\If ... 
\Then{
        \If ...
        \Then ...
        \Else ...
        \Fi
}\Else
\Fi
\end{verbatim}\end{rightside}

You are advised to place the extra braces exactly as above; don't let
extraneous spaces intervene between the keywords and the braces.

The \cs\If\ macro always starts a new line for the \kw{then} and
\kw{else} parts.  If you want \TeX\ to try to choose line breaks, use
\cs\SIf\ instead:

\begin{leftside}
  \begin{formula}
    \SIf a=b
    \Then c=d+e
    \Else p=q+r+s+t+u
    \Fi
  \end{formula}
\end{leftside}%
\begin{rightside}
\SIf a=b
\Then c=d+e
\Else p=q+r+s+t+u
\Fi
\end{verbatim}
\end{rightside}

\mbox{\kw{let}\dots\kw{in}} constructions are done in a similar way:
\cs\Let\ \mmexp \cs\In\ \mmexp, and \cs\SLet\ \mmexp
\cs\In\ \mmexp. 

\begin{leftside}
  \begin{formula}
    \Let x=f(y,z) \In
    g(x)+h(x)
  \end{formula}
\end{leftside}%
\begin{rightside}
\Let x=f(y,z) \In
g(x)+h(x)
\end{verbatim}
\end{rightside}
 
\begin{leftside}
  \begin{formula}
    \SLet x=f(y,z) \In{
      g(x)+h(x)
    }
  \end{formula}
\end{leftside}%
\begin{rightside}
\SLet x=f(y,z) \In{
g(x)+h(x)
}
\end{verbatim}
\end{rightside}

Notice that \cs\SLet\ takes a second argument, which is part of the
same `paragraph', where \cs\Let\ does not.

The typesetting of a \kw{cases} clause is more complicated.  It takes
the form:
\begin{verse}
\verb;\Cases{; \mmexp \verb;}; \\
from-\mmexp \verb;&; to-\mmexp \cs\\ \\
from-\mmexp \verb;&; to-\mmexp \cs\\ \\
\dots \\
\verb;\Otherwise{; \mmexp \verb;}; \\
\verb;\Endcases;
\end{verse}

The \cs\Otherwise\ field is optional.  This construction follows a
general pattern that is common in \Vdm\ input:  lists of things are
separated by \cs\\s, and subfields are separated by \verb;&;s or
\verb;:;s. 
\begin{dangerous}
In reality, there is another, optional argument, after the
\cs\Endcases.  If you were to try typesetting something like
\begin{verbatim}
        (... var = \Cases ...
                   \Endcases)
\end{verbatim}
you'd find the closing right parenthesis in an unexpected place (on
the same line as the $=$, in fact).  To get text to the right of the
\cs\Endcases\ you can place an optional argument within brackets
after it:
\begin{verbatim}
        (... var = \Cases ...
                   \Endcases[)]
\end{verbatim}
Admittedly, this looks a little strange, but it does work.
\end{dangerous}

Here is an example of \cs\Cases\ in action:

\begin{formula}
  \Cases{ select(x) }
  \nil            & \emptyset \\
  mk-Lst(hd,tl)  & \set{hd} \union elems(tl)
  \Otherwise{ x }
  \Endcases
\end{formula}

\begin{verbatim}
     \Cases{ select(x) }
     \nil            & \emptyset \\
     mk-Lst(hd,tl)  & \set{hd} \union elems(tl)
     \Otherwise{ x }
     \Endcases
\end{verbatim}

Note the \cs\\ is a {\em separator\/} and not a {\em
terminator\/}---you don't need one after the last item.  Also, the
\cs\Otherwise\ can appear anywhere between the \verb;\Cases{}; and the
\cs\Endcases, but it will always be typeset last.
\begin{dangerous}
Some people prefer the selectors to appear lined up on the left, some
on the right.  If you want them to appear on the left, say
\cs\leftCases; if you want them on the right, say
\cs\rightCases.  The scope of the \cs\leftCases\ and
\cs\rightCases\ commands is the current group.  By default, you
get \cs\rightCases.
\end{dangerous}

\subsection*{Other General Points about Formulas}

\cs\\ will%
\footnote{For `will' read `should'.} 
{\em always\/}
start a new line.  Sometimes this is done in addition to some other
function (as in the \cs\Cases\ macro, where it delimits a
particular case), but you should be able to use \cs\\ almost
anywhere to force a line break.  Indeed, sooner or later you'll want
to typeset a long formula and \TeX\ will not be able to break the line
sensibly, or will choose an unpleasant break.  In this case you'll
have to use \cs\\.

Frequently you need to indent things within multi-line formulas.  To
help you do this, a command is provided which breaks a line, and
indents the next line by an amount which you can supply (in units of
{\tt ems}).  The \cs\T\ command takes a single argument that controls
how much the next line will be indented:

\begin{leftside}
\begin{formula}
a \And b \T2
\Implies b \And a \T1
\Or d \And e
\end{formula}
\end{leftside}\begin{rightside}
a \And b \T2
\Implies b \And a \T1
\Or d \And e
\end{verbatim}\end{rightside}

The \cs\If, \cs\Let, etc., constructions are all unusual in
that it's impossible to typeset something sensibly to the right of
them.  For example, if you try
\begin{verbatim}
        \exists{x \in S}{
           \If x=0 \Then S=Q \Else S=P \Fi}
        \Or S=\emptyset
\end{verbatim}
then you'll get

\begin{vdm}
  \begin{formula}
    \exists{x \in S}{\If x=0 \Then S=Q \Else S=P \Fi} \Or S=\emptyset
  \end{formula}
\end{vdm}

\noindent which is unlikely to be what you wanted.

You should also remember that where \Vdm\ wants a \mmexp, \TeX\ will
be placed in math mode.  This is usually the right thing to do, but
occasionally you might want a natural language comment to appear
there.
In this case you'll have to insert an \cs\mbox\ or a \cs\parbox\
depending on whether your comment might span one or more lines:

\begin{leftside}
  \begin{formula}
    \If \mbox{the condition is true}
    \Then \mbox{do the true part}
    \Else "do the false part"
    \Fi
  \end{formula}
\end{leftside}%
\begin{rightside}
\If \mbox{the condition is true}
\Then \mbox{do the true part}
\Else "do the false part"
\Fi
\end{verbatim}\end{rightside}
The else-part illustrates how quotes can be used an an abbreviation
for \verb;\mbox{...}; within math mode.

Finally, all the constructions above will not break at a page
boundary.  This means that you're in big trouble if you want to
typeset a three-page \cs\Cases.  The only statement I can make to
mitigate this is: you shouldn't have expressions that complicated in
the first place---who do you expect to read them?  Remember: ``truth
is beauty'', so if your formulas are not beautiful, then chances are
they're not true either.





\section*{Typesetting data types}

The following table lists the primitive types and values available:

\begin{center}
\begin{tabular}{|l|l|l|}
\hline
$\set{0,1,\dots}$	& \Nat		& \cs\Nat 	\\
$\set{1,2,\dots}$	& \Natone	& \cs\Natone,\cs\Nati	\\
$\set{\dots,-1,0,1,\dots}$& \Int	& \cs\Int	\\
Real numbers		& \Real		& \cs\Real	\\
$\set{\true,\false}$	& \Bool		& \cs\Bool	\\
Truth			& \true		& \cs\true,\cs\True \\
Falsehood		& \false	& \cs\false,\cs\False\\
Nil			& \nil		& \cs\nil	\\
\hline
\end{tabular}
\end{center}

If you need a new keyword, you can create one easily.  For example, if
your favourite brand of logic has ``maybe'' as a value, you can say
\begin{verbatim}
        \makeNewKeyword{\maybe}{maybe}
\end{verbatim}
and henceforth \cs\maybe\ is a valid control sequence that produces
the text \kw{maybe}.  The text of the second argument to
\cs\makeNewKeyword\ can be anything; it doesn't have to match your
control sequence name.
\begin{dangerous}
If you don't like the font that keywords are set in, you can change
them by redefining the command \cs\keywordFontBeginSequence.  By
default it expands to \cs\small\cs\sf.
\end{dangerous}

The following type-related commands are provided:

\begin{center}
\begin{tabular}{|l|l|l|}
\hline
Output		& Input			& \\
\hline
$\setof{x}$	& \verb;\setof{x};	& set type constructor \\
$\set{a,b,c}$	& \verb;\set{a,b,c};	& set enumeration \\
$\emptyset$	& \cs\emptyset		& the empty set \\
$\seqof{x}$	& \verb;\seqof{x};	& seq. type constructor\\
$\seq{a,b,a,c}$	& \verb;\seq{a,b,a,c};	& seq. enumeration\\
$\emptyseq$	& \cs\emptyseq	& the empty sequence \\
$\mapof{x}{y}$	& \verb;\mapof{x}{y};	& map type constructor \\
$\mapinto{x}{y}$& \verb;\mapinto{x}{y};	& one-one map type \\
$\map{p\mapsto x}$
		& \verb;\map{p\mapsto x}; & map enumeration\\
$\emptymap$	& \cs\emptymap	& the empty map \\
\hline
\end{tabular}
\end{center}

\noindent Here are the relevant operators:

\begin{center}\small
\begin{tabular}{llllll}
$\in$		& \cs\in	&
	$\owr$	& \cs\owr	&
		$\sconc$	& \cs\sconc	\\
$\notin$	& \cs\notin	&
	$\dres$	& \cs\dres	&
		$\len{l}$	& \verb;\len{l};\\
$\subset$	& \cs\subset&
	$\rres$	& \cs\rres	&
		$\hd{l}$	& \verb;\hd{l};	\\
$\subseteq$	& \cs\subseteq&
	$\dsub$	& \cs\dsub	&
		$\tl{l}$	& \verb;\tl{l};	\\
$\inter$	& \cs\inter,\cs\intersection;&
	$\rsub$	& \cs\rsub	&
		$\elems{l}$	& \verb;\elems{l};\\
$\Inter$	& \cs\Inter,\cs\Intersection;&
	$\dom{m}$&\verb;\dom{m};&
		$\inds{l}$	& \verb;\inds{l};\\
$\union$	& \cs\union	&
	$\rng{m}$&\verb;\rng{m};&
				&		\\
$\Union$	& \cs\Union	&
	$\Min{s}$& \verb;\Min{s};&
				&	\\
$\diff$		& \cs\diff,\cs\difference;&
	$\Max{s}$& \verb;\Max{s};&	
				&	\\
$\card{s}$	& \verb;\card{s};&
		&		&
				&
\end{tabular}
\end{center}

\noindent
(\inds{} and \elems{} are not, strictly speaking, part of VDM any
more; use \dom{} and \rng{}.)

\begin{dangerous}
If you invent a new monadic keyword operator (like \dom{}, etc.),
then you can have \Vdm\ define for you a control sequence which
switches font, and puts the right spacing in.  For example,
\begin{verbatim}
        \newMonadicOperator{\inv}{inv}
\end{verbatim}
will define the \cs\inv\ control sequence to print
{\keywordFontBeginSequence inv\/}.  Henceforth you can say, e.g.,
\verb;\inv{Foo};.  All such sequences take one argument (they are
monadic, after all).
\end{dangerous}

You can define a new type using
\verb;\type{;type-name\verb;}{;type\verb;};:

\begin{leftside}
  \type{Complex}{\Real\x \Real}
\end{leftside}%
\begin{rightside}
\type{Complex}{\Real\x \Real}
\end{verbatim}\end{rightside}

Composites types can be typeset using the {\tt composite} environment:

\begin{leftside}
  \begin{composite}{Datec}
    day:\set{1,\ldots,366}, \\
    year:\set{1583,\ldots,2599}
  \end{composite}
\end{leftside}%
\begin{rightside}
\begin{composite}{Datec}
  day :\set{1,\ldots,366}, \\
  year:\set{1583,\ldots,2599}
\end{composite}
\end{verbatim}\end{rightside}

There is also a {\tt composite*} environment (and an equivalent
\cs\scompose\ control sequence) that places the entire composite
type on a single line:

\begin{leftside}
  \begin{composite*}{Celsius}
    \Real
  \end{composite*}
\end{leftside}%
\begin{rightside}
\begin{composite*}{Celsius}
  \Real
\end{composite*}
\end{verbatim}\end{rightside}

\begin{leftside}
  \scompose{Celsius}{\Real}
\end{leftside}%
\begin{rightside}
\scompose{Celsius}{\Real}
\end{verbatim}\end{rightside}

`Records' can be defined using the {\tt record\/} environment:

\begin{verse}
\verb;\begin{record}{;record-type-name\verb;}; \\
field-name \verb;:; field-type \cs\\ \\
\dots \\
\verb;\end{record};
\end{verse}
The colons are used as sub-field separators.

\begin{leftside}
  \begin{record}{PERSON}
    NM : \seqof{Char} \\
    FEM : \Bool
  \end{record}
\end{leftside}%
\begin{rightside}
    \begin{record}{PERSON}
      NM : \seqof{Char} \\
      FEM : \Bool
    \end{record}
\end{verbatim}\end{rightside}

If the definition is short, you may prefer to use a short form:
\begin{verbatim}
        \defrecord{PERSON}{
          NM : \seqof{Char} \\
          FEM : \Bool
        }
\end{verbatim}

\begin{dangerous}
Some people prefer the field names to appear lined up on the left, some
on the right.  If you want them to appear on the left, say
\cs\leftRecord; if you want them on the right, say
\cs\rightRecord.  The scope of the \cs\leftRecord\ and
\cs\rightRecord\ commands are the current group.  By default, you
get \cs\rightRecord.
\end{dangerous}

Updating fields of composites using the $\mu$-function can be
specified using \cs\chg:

\begin{leftside}
  \begin{formula}
    \chg{p}{FEM}{\Not man(q)}
  \end{formula}
\end{leftside}%
\begin{rightside}
\chg{p}{FEM}{\Not man(q)}
\end{verbatim}\end{rightside}

Notice that the $\mu$, parentheses, comma and $\mapsto$ were inserted
automatically.



\section*{How to Typeset Functions}

Typesetting $\lambda$-expressions is easy:

\begin{leftside}
  \begin{formula}
    \LambdaFn{x,y}{x^2+y^2}
  \end{formula}
\end{leftside}%
\begin{rightside}
\LambdaFn{x,y}{x^2+y^2}
\end{verbatim}\end{rightside}

As with \cs\forall, \cs\exists\ and \cs\unique,
\cs\LamdbaFn\ has a *-form that places the body of the function
below and to the right:

\begin{leftside}
  \begin{formula}
    \LambdaFn*{x,y,z}{
      (x^2+y^2+z^2)^{\frac12}}
  \end{formula}
\end{leftside}%
\begin{rightside}
\LambdaFn*{x,y,z}{
  (x^2+y^2+z^2)^{\frac12}}
\end{verbatim}\end{rightside}

There is also a {\tt fn\/} (function) environment for defining named
functions.  It has the following structure:
\begin{verse}
\verb;\begin{fn}{;name-of-function\verb;}{; argument-list \verb;}; \\
\verb;\signature{;signature-of-function\verb;}; \\
\^{optional precondition}\\
\^{optional postcondition}\\
 body of function (a \mmexp) \\
\cs\end{fn}
\end{verse}

See the first page for an example.  The \cs\signature\ is optional
and can be placed anywhere within the body---it will always be typeset
before the body.  Useful macros within the \cs\signature\ are:
\cs\x\ and \cs\to, which yield $\x$ and~$\to$.  Note that you can also
enter functions defined implicitly with pre- and post-conditions; see the
next section on how to enter them.

All of the material in the section on formulas is relevant within the
body of the function.

\sloppy\sloppy
If you frequently intersperse your function definitions with text (and you
should), you can save some typing by using the {\tt vdmfn\/} environment.
\cs\begin\verb;{vdmfn}; \dots \cs\end\verb;{vdmfn}; is equivalent to
\cs\begin\verb;{vdm};\cs\begin\verb;{fn}; \dots
\cs\end\verb;{fn};\cs\end\verb;{vdm};.

The {\tt fn} environment also has a *-form that omits inserting
parentheses around the argument list.  For example:


\begin{leftside}
\begin{fn*}{MP}{\term{p}\rho\sigma}
\ldots
\end{fn*}
\end{leftside}\begin{rightside}
\begin{fn*}{MP}{
  \term{p}\rho\sigma}
...
\end{fn*}
\end{verbatim}\end{rightside}

If you require the $\DEF$ symbol by itself (for example, in a
footnote), then you can get it by saying \cs\DEF.

\subsection*{Invariants}

To typeset an invariant on a composite object, use the following
structure:

\begin{leftside}
\begin{record}{D}
  day : Day \\
  year : Year
\end{record}
\where
\begin{fn}{inv-D}{mk-D(d,y)}
  is-leapyr(y) \Or d \le 365
\end{fn}
\end{leftside}\begin{rightside}
\begin{record}{D}
  day : Day \\
  year : Year
\end{record}
\where
\begin{fn}{inv-D}{mk-D(d,y)}
  is-leapyr(y) \Or d \le 365
\end{fn}
\end{verbatim}\end{rightside}


\section*{How to Typeset Operations}

Operations are typeset within the {\tt op\/} environment.
The general structure is:

\begin{verse}
\verb;\begin{op}{;\^{name-of-operation}\verb;}; \\
\verb;\args{;\^{list-of-arguments}\verb;}; \\
\verb;\res{;\^{result(s)}\verb;}; \\
\verb;\ext{;\^{list-of-externals}\verb;}; \\
\^{pre-condition} \\
\^{post-condition} \\
\verb;\end{op};
\end{verse}

Any of \cs\args, \cs\res, \cs\ext, \^{pre-condition} or
\^{post-condition} may be omitted.  \verb;\begin{vdmop}; is an
abbreviation for \verb;\begin{vdm}\begin{op};;  \verb;\end{vdmop}; is an
abbreviation for \verb;\end{op}\end{vdm};.

The \^{name-of-operation} can be any one-line expression; it is
typeset in math mode.

The \^{list-of-arguments} is a \mmexp\ that can span multiple lines;
force a newline with \cs\\.  If present it is placed within
parentheses.

The \^{result(s)} is also any \mmexp.  It is typeset to the right of
any arguments.

The \^{list-of-externals} takes the following form:
\begin{verse}
\verb;\ext{;	\\
\quad $\langle$optional \cs\Rd\ or \cs\Wr$\rangle$
	\^{external-name(s)} {\tt :\/} \^{external-types} \cs\\ \\
\quad $\langle$optional \cs\Rd\ or \cs\Wr$\rangle$
	\^{external-name(s)} {\tt :\/} \^{external-types} \cs\\ \\
\dots \\
\verb;};
\end{verse}
Alternatively, if the list of externals is long (say, more than five
lines) the {\tt externals\/} environmment can be used:
\begin{verse}
\verb;\begin{externals};	\\
\quad $\langle$optional \cs\Rd\ or \cs\Wr$\rangle$
	\^{external-name(s)} {\tt :\/} \^{external-types} \cs\\ \\
\quad $\langle$optional \cs\Rd\ or \cs\Wr$\rangle$
	\^{external-name(s)} {\tt :\/} \^{external-types} \cs\\ \\
\dots \\
\verb;\end{externals};
\end{verse}
\begin{dangerous}
Some people prefer the externals identifiers to appear lined up on the
left, some on the right.  If you want them to appear on the left, say
\cs\leftExternals; if you want them on the right, say
\cs\rightExternals.  The scope of the \cs\leftExternals\ and
\cs\rightExternals\ commands are the current group.  By default,
you get \cs\leftExternals.
\end{dangerous}


The \^{pre-condition} and \^{post-condition} take similar forms:
\begin{verse}
\verb;\pre{;\mmexp\verb;}; \\
\end{verse}
or
\begin{verse}
\cs\begin{precond} \\
\mmexp \\
\verb;\end{precond};
\end{verse}
and
\begin{verse}
\verb;\post{;\mmexp\verb;}; \\
\end{verse}
or
\begin{verse}
\verb;\begin{postcond}; \\
\mmexp \\
\verb;\end{postcond};
\end{verse}
Use the \cs\begin\dots\cs\end\ style if the \mmexp\ is longer
than a few lines.
All of the constructs mentioned in the section on formulas can be used
within pre- and post-conditions.

\section*{Proofs}

Here's an example of typesetting proofs in the ``natural deduction''
style. 

\begin{proof}
  \From E@1 \Or E@2 	\\
1   \From E@1	\\
    \Infer E@2 \Or E@1	\` $\vee$-I(h1) \\
2   \From E@2		\\
    \Infer E@2 \Or E@1	\` $\vee$-I(h2) \\
  \Infer E@2 \Or E@1	\` $\vee$-E(h,1,2)\\
\end{proof}
\begin{verbatim}
        \begin{proof}
            \From E@1 \Or E@2                           \\
        1       \From E@1                               \\
                \Infer E@2 \Or E@1  \by $\vee$-I(h1)    \\
        2       \From E@2                               \\
                \Infer E@2 \Or E@1  \by $\vee$-I(h2)    \\
            \Infer E@2 \Or E@1      \by $\vee$-E(h,1,2) \\
        \end{proof}
\end{verbatim}

Proofs are embedded within the {\tt proof} environment.  (A proof does
not have to be within a {\tt vdm} environment.)  Each line of the
proof ends with \cs\\.  Lines that begin a subproof
have \cs\From\ after the equation number.  Lines that end a
subproof have \cs\Infer\ after the equation number.  Other lines
have \cs\& after the equation number (see next example).  If you
don't need an equation number, just omit it,  but you must have one of
either \cs\From, \cs\Infer\ or \cs\& on each proof line.
If you want to include a justification of a particular proof line at
the right hand end of the line, type it after a \cs\by.  \cs\by\
is optional; you needn't include it if you don't need a justification.

Points worth bearing in mind:
\begin{itemize}
\item	You are automatically placed in math mode after the
	\cs\From, \cs\Infer\ or \cs\&; the math mode ends
	at the next \cs\by\ or \cs\\.
\item	You {\em cannot} break a line in the middle simply by using
	\cs\\ before \cs\by; you must use separate proof lines
	to split a formula.
\item	You are within a {\tt tabbing} environment within a proof, so
	you can use all the usual {\tt tabbing} commands (\cs\=,
	\cs\>, etc.) to line things up across proof lines.  Note
	that you will explicitly have to enter math mode again after
	any of these commands though.
\end{itemize}

Here's another example:

\begin{proof}
  \From \forall{x\in X}{E(x); s\in X}	\\
1 \&	\Not\exi