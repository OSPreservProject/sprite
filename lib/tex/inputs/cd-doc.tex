%%%%%%%%%%%%%%%%%%%%%%%%%%%%%%%%% Cut Here %%%%%%%%%%%%%%%%%%%%%%%%%%%%%%%%%%%%
% cddoc.tex
\documentstyle[11pt,cd]{article}
\nofiles

\addtolength{\textwidth}{1in}
\addtolength{\oddsidemargin}{-.5in}

\newcommand{\AmSTeX}{$\cal A$\kern-.1667em\lower.5ex\hbox
 {$\cal M$}\kern-.125em{$\cal S$}-\TeX}

\newcommand{\cdrl}{\cd\rightleftarrows}
\newcommand{\cdlr}{\cd\leftrightarrows}
\newcommand{\cdr}{\cd\rightarrow}
\newcommand{\cdl}{\cd\leftarrow}
\newcommand{\cdu}{\cd\uparrow}
\newcommand{\cdd}{\cd\downarrow}
\newcommand{\cdud}{\cd\updownarrows}
\newcommand{\cddu}{\cd\downuparrows}


\begin{document}\thispagestyle{empty}
\begin{center}
{\Large Commutative Diagrams for \LaTeX} \\
March 3, 1989
\end{center}

\paragraph{Commutative diagrams} These were adapted from those in \AmSTeX\
(see p.\ 146 of {\em The Joy of \TeX}). All of the horizontal ``arrows'' will
stretch with the superscripts and subscripts. These can also be used outside
of \verb"\CD", where they will be somewhat shorter. You must put `{\tt cd}'
in with the other style options, for example
\verb"\documentstyle[12pt,cd]{article}".  The \verb"@"-commands from the
\AmSTeX\ still work, but the new \verb"\cd"-style is the preferred format.
The table below shows the arrows and names, along with suggested
abbreviations.

\begin{center}
\begin{tabular}{|c|c|c||c|c|c|} \hline
Arrow & Name & Abbr. & Arrow & Name & Abbr.\\ \hline
$\rightarrow$ & \verb"\cd\rightarrow", \verb"\cd>" & \verb"\cdr" &
        $\leftarrow$ & \verb"\cd\leftarrow", \verb"\cd<" & \verb"\cdl" \\
$\cd\rightleftarrows {}{}$ &\verb"\cd\rightleftarrows" &\verb"\cdrl" &
        $\cd\leftrightarrows{}{}$ &\verb"\cd\leftrightarrows"
        &\verb"\cdlr" \\
$\uparrow$ &\verb"\cd\uparrow" & \verb"\cdu" &
        $\downarrow$ &\verb"\cd\downarrow" &\verb"\cdd" \\
$\uparrow\downarrow$ &\verb"\cd\updownarrows" & \verb"\cdud" &
        $\downarrow\uparrow$ &\verb"\cd\downuparrows" &\verb"\cddu" \\
        $|$ & \verb"\cd|" && $\|$ & \verb"\cd\|" & \\
$\cd={}{}$ & \verb"\cd=" && None & \verb"\cd." &  \\
\hline
\end{tabular}
\end{center}

\paragraph{Example}
\begin{center}
\begin{tabular}{l}
\verb"$$\CD" \\
\verb"G \cdrl {\gamma}{\delta}  H  \cdr {}{\Delta} K \\" \\
\verb"\cd.  \cdud {f}{g}  \cd| {h}{k} \\" \\
\verb"0 \cdr {}{}  G'  \cd= {\beta}{}  H'"
\verb"\endCD $$"
\end{tabular}
\hfil
$\CD
G  \cdrl {\gamma}{\delta}  H  \cdr {}{\Delta} K \\
\cd. \cdud {f}{g}       \cd| {h}{k} \\
0 \cdr {}{} G' \cd= {\beta}{}   H'
\endCD$
\end{center}


\bigskip\noindent
The abbreviations were made with the following \verb"\newcommand"'s at the
top of the file:
\begin{verbatim}
\newcommand{\cdrl}{\cd\rightleftarrows}
\newcommand{\cdlr}{\cd\leftrightarrows}
\newcommand{\cdr}{\cd\rightarrow}
\newcommand{\cdl}{\cd\leftarrow}
\newcommand{\cdu}{\cd\uparrow}
\newcommand{\cdd}{\cd\downarrow}
\newcommand{\cdud}{\cd\updownarrows}
\newcommand{\cddu}{\cd\downuparrows}
\end{verbatim}

\bigskip
\begin{center}
Darrel Hankerson (Bitnet: \verb"hank@auducvax") \\
Algebra, Combinatorics, \& Analysis \\
Auburn University \\
Auburn, Alabama 36849
\end{center}

