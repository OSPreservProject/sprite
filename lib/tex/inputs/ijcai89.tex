% This is the instructions for authors for IJCAI'89.
\documentstyle[ijcai89,named]{article}
% The file ijcai89.sty is the style file for IJCAI'89.
% The file named.sty contains macros for named citations as produced 
% by named.bst.

\title{IJCAI'89 Format Instructions for Authors \\ (Computer-Typeset Papers)}
\author{Shirley Jowell\thanks{Acknowledgments to sponsoring agencies
		appear here as a footnote.}\\ 
	Morgan Kaufmann Publishers \\
	2929 Campus Drive, Suite 260 \\
	San Mateo, California \hspace{1em} 94403
	\And
	Peter F. Patel-Schneider \\
	AT\&T Bell Laboratories \\
	600 Mountain Avenue \\
	Murray Hill, New Jersey \hspace{1em} 07974
	}

\newcommand{\usa}[2]{#1}

\begin{document}

\maketitle

\begin{abstract}
The {\it IJCAI'89 Proceedings} will be printed using the photo-offset print
process directly from camera-ready copy furnished by the authors.  To
ensure that all papers in the {\it Proceedings} have a uniform appearance,
authors are asked to adhere to the following instructions.  
\end{abstract}

\section{Introduction}

Papers produced for IJCAI'89 using computer typesetting systems (such as
\LaTeX{} or Scribe) will be printed directly from {\bf $8$-$1/2 \times
11''$} masters.

Print your paper using a laser or other letter-quality printer in good
condition.  (If you have a ``write-white'' laser printer you must use fonts
specifically designed for it---not fonts for ``write-black'' printers.)
{\em Do not use line printers or dot matrix printers.} Papers with poor
quality output, e.g., light or gray type, and papers that significantly
deviate from these instructions, e.g., by using 8-point fonts, will not be
included in the proceedings.

Print your paper on heavy bond white paper.  Do not fold or crease your paper.

\subsection{Typewritten Papers}

If you do not have access to a typesetting system, you can type your paper
on special, oversize forms (``large mats'') that will be reduced before
printing by Morgan Kaufmann Publishers.  

\usa{If you must type your paper, please contact
Shirley Jowell at Morgan Kaufmann for a supply of mats and a
separate set of instructions for typing your paper.  {\em Discard these
instructions\/ {\em (which are for computer-typeset papers only)} if you
are typing your paper on large mats.}}{Included in this mailing is a set of
large mats that should be adequate to handle your paper; also included is a
separate set of instructions for using the mats.  If you type your paper,
{\em please discard these instructions\/} (which are for computer-typeset
papers only.)}

\section{Style and Format}

\LaTeX{} and Bib\TeX{} style files that implement these instructions have
been placed in the \LaTeX{} style repository.  (See Appendix~\ref{latex} for
instructions on how to obtain these files.)

\subsection{General Instructions}

Print manuscripts two columns to a page, in the manner in which these
instructions are printed.  The exact dimensions for pages are:
\begin{itemize}
\item left and right margins: $.75''$,
\item column width: $3.375''$,
\item gap between columns: $.25''$,
\item top margin---first page: $1.25''$,
\item top margin---other pages: $.625''$,
\item bottom margin: $1.0''$.
\end{itemize}
All measurements assume an {\bf $8$-$1/2 \times 11''$} page size.

A format template is included with these instructions.

Use 10-point type in a clear, readable font with 1-point leading (10
on 11).  Indent when starting a new paragraph, except after major headings.

\subsection{Title Page}

\subsubsection{Title and Author Information}

Center the title on the entire width of the page in a 14-point bold font.
Place the names of authors below the title in a 12-point bold font and
affiliations and complete addresses directly below the author names in a
12-point (non-bold) font.

Credit to a sponsoring agency appears in a footnote at the bottom of the
left column of the first page.  See the example on this page.

\subsubsection{Abstract}

Place the abstract at the beginning of the first column $3.0''$ from the
top of the page, unless that does not leave enough room for the title and
author information.  Use a slightly smaller width than in the body of the
paper.  Head the abstract with ``Abstract'' centered above the body of the
abstract in a 12-point bold font.

Abstracts should be no longer than 200 words.

\subsection{Text}
The main body of the text immediately follows the Abstract. 

\subsection{Sections}

Please note that the following section heading sizes are not the \LaTeX{}
standard.

\subsubsection{Section Headings}

Print section headings in 12-point bold type in the style shown in these
instructions.  Leave a blank space of approximately 10 points above and 8
points below section headings.

\subsubsection{Subsection Headings}

Print subsection headings in 10-point bold type.  Leave a blank space of
approximately 9 points above and 4 points below subsection headings.

\subsubsection{Subsubsection Headings}

Print subsubsection headings in 10-point bold type.  Leave a blank space of
approximately 8 points above and 3 points below subsubsection headings.

\subsubsection{Special Sections}

Arrange special sections as follows:
\begin{description}
\item[Acknowledgments:] 
The Acknowledgments section, if included, follows the main body of the text
and is headed ``Acknowledgments,'' printed in the same style as a section
heading, but without a number. 
\item[Appendices:] 
Any Appendices follow the Acknowledgments (or directly follow the text) and
look like sections, except that they are numbered with capital letters
instead of arabic numerals. 
\item[References:]
The References section is headed ``References,'' printed in the same
style as a section heading, but without a number.
A sample list of references is given at the end of these
instructions.
Use a consistent format for references, such as provided by
Bib\TeX{}.
\end{description}

\subsection{Citations}

Citations within the text should include the author's last name and
the year of publication, for example \cite{cheeseman:probability}.
Append lower case letters to the year in cases of ambiguity.
Treat multiple authors as in the following examples:
\cite{abelson-et-al:scheme} (for more than two authors) and
\cite{brachman-schmolze:kl-one} (for two authors).
If the author portion of a citation is obvious, omit it,
e.g., Levesque \shortcite{levesque:belief}.
Collapse multiple citations as follows:
\cite{levesque:functional-foundations,haugeland:mind-design}.%
\nocite{abelson-et-al:scheme}%
\nocite{brachman-schmolze:kl-one}%
\nocite{cheeseman:probability}%
\nocite{haugeland:mind-design}%
\nocite{lenat:heuristics}%
\nocite{levesque:functional-foundations}%
\nocite{levesque:belief}

\subsection{Footnotes}

Place footnotes at the bottom of the page in a 9-point font.  Refer to them
with superscript numbers.\footnote{This is how your footnotes should
appear.} Separate them from the text by a short line.\footnote{Note the
line separating these footnotes from the text.}


\section{Illustrations}

\subsection{General Instructions}

Place illustrations (figures, drawings, tables, and photographs) throughout
the paper at the places where they are first discussed, rather than at the
end of the paper.  If placed at the bottom or top of a page, illustrations
may run across both columns.  Securely attach them to the master form with
glue stick, spray adhesive, rubber cement, or write tape.  Do not use
transparent tape as the printing process blurs copy under transparent tape.

Number illustrations sequentially.  Use references of the following form:
Figure 1, Table 2, etc.  Place illustration numbers and captions under
illustrations.  Leave a margin of 1/4-inch around the area covered by the
illustration and caption.  Use 9-point type for captions, labels, and
other text in illustrations.

{\em Do not use line printer printouts in illustrations.}

\subsection{Drawings}

Draw original line drawings in {\em black ink}, not pencil.  Do not color
in drawings.  Lines should be heavy enough to reproduce clearly.

\subsection{Photographs}

Use only glossy black and white photographs.  Color photographs do not
reproduce well.  (Red will reproduce as black, for example.)  Photographs
incur extra expense, so please use them judiciously.

\section{Length of Papers}

Authors are allowed {\em six\/} (6) pages.  All illustrations and
references must be included in the 6-page allowance.  

{\em One} extra page may be included by accompanying the paper with a check
for \$250, payable to the American Association for Artificial Intelligence.
Papers over seven pages will not be accepted for publication.

\section{Identification}

Type or write your name on the {\em back\/} of every page of the masters.
Number the pages sequentially.  This information is for identification only;
final page numbers will be assigned by the publisher.

\section{Mailing}

{\em Make a photocopy of your final paper.}  Keep the photocopy in your
files for reference or in case the original is lost in the mail.

Your paper (the original) must be {\em received\/} by Morgan Kaufmann
Publishers no later than {\bf May 1, 1989}.  {\em Papers received later
than this date will not be included in the Proceedings.}

Do not fold your paper for mailing.  Please use the enclosed envelope for
mailing to Morgan Kaufmann.  If you use a different envelope, mark it
clearly: {\bf Do Not Fold or Bend}.

Send to:
\begin{quote}
Shirley Jowell \\
Attn: IJCAI'89 Conference \\
Morgan Kaufmann Publishers \\
2929 Campus Drive, Suite 260 \\
San Mateo, CA \hspace{1em} 94403\\
U. S. A.
\end{quote}

\section{Inquiries}

If you have any questions about the preparation or submission of your
paper, please contact:
\begin{quote}
Shirley Jowell \\
Morgan Kaufmann Publishers \\
(415) 578-9911
\end{quote}

\appendix


\section{Using \LaTeX{}}\label{latex}

A \LaTeX{} style file (for version 2.09 of \LaTeX{}) that implements these
instructions has been prepared, as have a Bib\TeX{} style file (for version
0.99c of Bib\TeX{}, {\em not version 0.98i}) and a LaTeX style file that
implement the citation and reference styles in these instructions.

The relevant files have been placed in the \LaTeX{} style collection at
Clarkson.  To retrieve these files use one of the following methods:

\begin{enumerate} 
\item For Internet users - how to ftp:

Here is an example session.  Ftp syntax varies from host to
host; your syntax may be different.

\begin{verbatim}
% ftp sun.soe.clarkson.edu
Name: anonymous
Password: <any non-null string>
ftp> cd pub/latex-style
ftp> get ijcai89.tex
ftp> get ijcai89.sty
ftp> get named.sty
ftp> cd ../bibtex-style
ftp> get named.bst
ftp> quit
\end{verbatim}

\item  Non-Internet users - how to retrieve by mail:

To retrieve files documentation send mail to
archive-server@sun.soe.clarkson.edu as follows:

\begin{verbatim}
To: archive-server@sun.soe.clarkson.edu
Subject:

path <your address>
send latex-style ijcai89.tex
send latex-style ijcai89.sty named.sty
send bibtex-style named.bst
\end{verbatim}

The address you put in the message should be a valid internet path,
such as host!user@uunet.uu.net.

\end{enumerate}

Further information on using these styles for the preparation of papers for
IJCAI'89 can be obtained by writing to
\begin{quote}
Peter F. Patel-Schneider \\
AT\&T Bell Laboratories \\
600 Mountain Ave., Room 3C-410A \\
Murray Hill, New Jersey \hspace{1em} 07974 \\
U. S. A. \\
pfps@research.att.com \\
(201) 582-3399
\end{quote}

%% This section was initially prepared using BibTeX.  The .bbl file was
%% placed here later
%\bibliography{publications}
%\bibliographystyle{named}
%% The file named.bst is a bibliography style file for BibTeX 0.99c
\begin{thebibliography}{}

\bibitem[\protect\citeauthoryear{Abelson \bgroup \em et al.\egroup
  }{1985}]{abelson-et-al:scheme}
Harold Abelson, Gerald~Jay Sussman, and Julie Sussman.
\newblock {\em Structure and Interpretation of Computer Programs}.
\newblock MIT Press, Cambridge, Massachusetts, 1985.

\bibitem[\protect\citeauthoryear{Brachman and
  Schmolze}{1985}]{brachman-schmolze:kl-one}
Ronald~J. Brachman and James~G. Schmolze.
\newblock An overview of the {KL-ONE} knowledge representation system.
\newblock {\em Cognitive Science}, 9(2):171--216, April--June 1985.

\bibitem[\protect\citeauthoryear{Cheeseman}{1985}]{cheeseman:probability}
Peter Cheeseman.
\newblock In defense of probability.
\newblock In {\em Proceedings of the Ninth International Joint Conference on
  Artificial Intelligence}, pages 1002--1009, Los Angeles, California, August
  1985. International Joint Committee on Artificial Intelligence.

\bibitem[\protect\citeauthoryear{Haugeland}{1981}]{haugeland:mind-design}
John Haugeland, editor.
\newblock {\em Mind Design}.
\newblock Bradford Books, Montgomery, Vermont, 1981.

\bibitem[\protect\citeauthoryear{Lenat}{1981}]{lenat:heuristics}
Douglas~B. Lenat.
\newblock The nature of heuristics.
\newblock Technical Report CIS-12 (SSL-81-1), Xerox Palo Alto Research Centers,
  April 1981.

\bibitem[\protect\citeauthoryear{Levesque}{1984a}]{levesque:functional-foundat%
ions}
Hector~J. Levesque.
\newblock Foundations of a functional approach to knowledge representation.
\newblock {\em Artificial Intelligence}, 23(2):155--212, July 1984.

\bibitem[\protect\citeauthoryear{Levesque}{1984b}]{levesque:belief}
Hector~J. Levesque.
\newblock A logic of implicit and explicit belief.
\newblock In {\em Proceedings of the Fourth National Conference on Artificial
  Intelligence}, pages 198--202, Austin, Texas, August 1984. American
  Association for Artificial Intelligence.

\end{thebibliography}



\end{document}


