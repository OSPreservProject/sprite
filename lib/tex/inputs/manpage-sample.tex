\documentstyle[11pt,twoside,manpage]{report}
\begin{document}

\date{November 25, 1990}

\begin{titlepage}
    \vspace*{1.0in}
    \begin{flushright}
          {\huge\bf OIM C++ Library} \\[.25cm]
          {\rule[.1in]{4.5in}{.02in}}\\[.25cm]
          {\large\bf Reference Manual} \\[2.0cm]
          {\large\bf Version 1.0} \\[4.0cm]
          {\bf Rong Chen}\\
          {\bf Junsheng Long}\\[1.0cm]
          {Research Services}\\
          {Office for Information Management}\\
          {1407 W. Gregory Drive}\\
          {Urbana, IL 61801}\\
          {(217)--244--8538}\\
    \end{flushright}
\end{titlepage}

\pagenumbering{roman}
\tableofcontents

\chapter*{Preface}

The Object-oriented problem solving and object-oriented programming 
represent a way of thinking and a methodology for computer
programming that are quite different from the traditional approaches 
supported by structured programming languages. The powerful 
features of object-oriented programming support the concepts that 
make computer problem solving a more human-like activity and 
that increase the re-usability of software code. As object-oriented
languages are steadily available, object-oriented programming
becomes more and more popular among both system developers 
and application programmers.

C++ is a hybrid language that fuses object-oriented functionality
with the features of a traditional and efficient structured language, C. 
C++ provides programmers and problem solvers object-oriented capability
without loss of run-time or memory efficiency. In addition,
C++ is available in almost every computer systems from PC to mainframe.

\newpage

\raggedbottom
\pagenumbering{arabic}
\pagestyle{headings}
\chapter*{Class Descriptions}

\begin{manpage}{OIM C++ Library}{Choice}{Version 1.0}

\subtitle{Name}
    Choice --- a set of boxes representing the choices available. 

\subtitle{Declaration}
    \#include \<choice.h\>

\function{Choice(Point origin, Point corner,
		int foreground_color, int background_color,
		int frame_style, int space_in_between,
                int number_of_choices,
                const~char\* first_label, $...$ )}
    Constructs a choice object.  Each choice object is a set of boxes
    representing the choices available.  "Origin" and "corner" define the
    rectangular area inside which the boxes are displayed. 
    The "space_in_between" (in pixels)
    defines the distance between two adjacent boxes.  Depending on 
    whether there is enough room to put all the choices horizontally
    or vertically in the designated area,
    the choices will be laid out accordingly.  The
    attributes "foreground_color", "background_color" and "frame_style"
    define the colors for each box that represents each choice.
    The number of labels starting with "first_label" must match the
    number given by "number_of_choices".  Each label is displayed within
    its corresponding box.  The choices are displayed 
    as soon as they are created.

\function{Choice(int number_of_choices, Boolean is_horizontal = VERTICAL)}
    Constructs an array of choices or 
    a choice object with boxes of different sizes.  The attributes
    of the actual choices will be defined later using the
    addItem() operation. Note: It is forbidden to have arguments
    in the constructors of array objects in C++. 

\function{\~Choice()}
    Choice class has an empty destructor.

\subtitle{Description}
    A Choice object is a set of related boxes. It is used to display
    multiple choices on the screen. 
    A choice may either be selected by pressing a hot key
    (when mouse is not enabled) or by pressing a mouse button when the mouse
    cursor is within the intended choice box.  
    The first highlighted character in the text label will be the 
    hot key if the mouse is not enabled, otherwise, the highlighted
    character will have no effect. For example, label "@N@ext" designates
    "N" as the hot key for the button if the mouse is not enabled.
    By default, if the mouse driver is loaded before program execution,
    the mouse is enabled.  When using the mouse to select
    a choice, the selection is immediate; a hot key must be confirmed
    by pressing ENTER (carriage return key).  Arrow keys can also be used
    to move the current selection in the direction indicated by the arrows,
    this also requires an ENTER to confirm.

\subtitle{Global \\ Variables}
\variable{char host_name[20] = \"rose.cs.uiuc.edu\"}
    The "host_name" buffer is initialized to the current machine name
    before the main() function is executed.  The machine name is
    usually declared in a DOS shell variable "OIMUSER".

\variable{Boolean mouse_installed}
    By default, the "mouse_installed" variable is assigned to TRUE if
    the mouse driver has been installed before program execution;
    otherwise it is assigned to FALSE.  The value may also be changed
    through enableMouse() and disableMouse() functions.

\subtitle{Macros}
\function{MAX({\it x}, {\it y})}
    Returns the maximum of the two comparable objects "x", and "y".
    This macro is often used to compare integers or real numbers.

\function{MIN({\it x}, {\it y})}
    Returns the minimum of the two comparable objects "x", and "y".
    This macro is often used to compare integers or real numbers.

\subtitle{Public \\ Operations}
\function{void addItem(Point origin, Point corner,
		 int foreground_color, int background_color,
		 int frame_style, const~char\* label,
		 int text_color = WHITE,
		 int text_highlight_color = LIGHT_WHITE)}
    Defines the attributes of a choice box in the array declared by the
    Choice(int "number_of_choices", Boolean "is_horizontal" = VERTICAL) 
    constructor. It must be called exactly "number_of_choices"
    times.  The first call specifies the first choice, second call specifies
    the second choice, and so on.  The attributes "origin" and "corner"
    define a box area where the choice item is located on the screen.  Other
    attributes are self-explanatory.  This call is particularly useful
    on program setup pages.

\function{Boolean hasValue()}
    Returns TRUE if the designated choice object has been selected at
    least once.  Graphically,
    it means one of the choices is rendered in reverse colors.  It is
    sometimes necessary to check if all choice objects on a screen have
    been selected, so that a program can go on to the next page of screen.
    If a previous select() call is returned abnormally by pressing an ESC key,
    this function will return FALSE. Should this occur, the previous 
    value, if any, will also be lost.

\subtitle{Virtual \\ Operations}

\function{void action(int status)}
    Defines the behaviors of the choice object when a choice handler
    has been bonded to the object.  This function is called by the
    event handler when it detects that the choice object has been
    selected.  The default action is to do nothing if no choice handler
    presents, otherwise, it invokes the choice handler 
    and passes it along with the current status 
    (whatever has been returned from a status() call).  
    Do not modify this function unless it is absolutely necessary because
    it redefines the semantics of all choice objects.

\function{void bind(void \hbox{\rm (\*{\it handler})(int)})}
    Associates a choice handler function with the choice object.  Every
    time this choice object is invoked, the choice handler is called
    immediately.  
    The handler is passed along with the current "status"
    (return value of status() call) as its argument. The default 
    action taken by the event handler can be modified by changing
    the next function action().

\function{Boolean isSelected(Point cursor)}
    Returns TRUE if a legal hot key is pressed for the
    the indicated choice object. It also returns TRUE if 
    a mouse button is pressed within the range of the choice object area.
    This call is used by the event handler to check if this choice object
    has been selected.  It is hardly needed in regular applications.
    The "cursor" is for mouse cursor in mouse mode or faked with the
    current keyboard input as the cursor's x coordinate.

\subtitle{See Also}
     Event, Button, Box

\subtitle{Author}
    Ron Chen at the Office for Information Management, UIUC. 9-1-89.

\end{manpage}
\newpage
\begin{verbatim}
    /* EXAMPLE 2 */
    #include <stdlib.h>
    #include "choice.h"

    void handler1(int pick) {
        Box box(500, 30, 600, 60, LIGHT_CYAN, LIGHT_MAGENTA);
        box << pick + 1;
    }

    void handler2(int pick) {
        Box box(500, 30, 600, 60, LIGHT_CYAN, LIGHT_MAGENTA);
        box << (pick==0 ? "Ape" : (pick==1 ? "Bob" : "Car"));
    }

    void handler3(int pick) {
        Box box(500, 30, 600, 60, LIGHT_CYAN, LIGHT_MAGENTA);
        switch (pick) {
            case 0: box << "D"; break; case 1: box << "E"; break;
            case 2: box << "F"; break; case 3: box << "G"; break;
            case 4: exit(0);
        }
    }

    main() {
        Point a(100, 100), b(220, 130);
        Choice c1(a, b, BROWN, RED, ON_BORDER, 0, 4,
                       "@1@", "@2@", "@3@", "@4@");
        Choice c2(3, HORIZONTAL);
        c2.addItem(Point(200, 200), Point(260,230),
                   WHITE, BLUE, IN_BORDER, "@A@pe");
        c2.addItem(Point(270, 200), Point(330,230),
                   WHITE, BLUE, IN_BORDER, "@B@ob");
        c2.addItem(Point(340, 200), Point(400,230),
                   WHITE, BLUE, IN_BORDER, "@C@ar");
        a = Point(500, 230); b = Point(550, 330);
        Choice c3(a, b, GREEN, GRAY, NO_BORDER, 10, 5,
                       "@D@", "@E@", "@F@", "@G@", "@H@");
        c1.bind(handler1);
        c2.bind(handler2);
        c3.bind(handler3);
        Event::waitForEvents();
    }
\end{verbatim}

\end{document}
