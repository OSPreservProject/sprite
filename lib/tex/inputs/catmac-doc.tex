%Please reply to inhb@mcgillb.bitnet.
%For this semester only, to barr@linc.cis.upenn.edu.
%Michael Barr
%Math Dept
%McGill Univ.
%805 Sherbrooke St W
%Montreal, QC
%Canada H3P 1S4
\documentstyle[12pt,catmac]{article}
\textheight=9in \topmargin=0pt\headheight=0pt\headsep=0pt
\textwidth=6.5in \oddsidemargin=0pt
\begin{document}
{\catcode`\ =13\global\let =\ \catcode`\^^M=13
\gdef^^M{\par\noindent}}
\def\verbatim{\tt
\catcode`\^^M=13
\catcode`\ =13
\catcode`\\=12
\catcode`\{=12
\catcode`\}=12
\catcode`\_=12
\catcode`\^=12
\catcode`\&=12
\catcode`\~=12
\catcode`\#=12
\catcode`\%=12
\catcode`\$=12
\catcode`|=0
}

All commercial rights reserved.  May be freely distributed
and used with the following exceptions:\\
1. No commercial use without explicit permission.\\
2. It may not be used by any employee of a telephone company.\\
3. It may not be distributed without this notice.

\def\\{{\tt \char 92}}

\section*{The catmac macros}
The file catmac includes a number of macros for drawing commutative
diagrams in special shapes as well as more flexible macros to paste
together more complicated diagrams.

In general, there are two kinds of macros.  The first kind has the
following form
{\tt \\shapename[shape\_parameters]} and is intended to be used in the
following way:
{\verbatim
\begin{center}
\resetparms
\shapename[shape_parameters]
\end{center}
|egroup

This will create a shape with the given parameters.  It operates by
building its own picture environment.

The second kind of procedure must be used inside a picture environment
and allows the user to put a diagram of a given size and shape at a
given place.  The usual way in which this is used is:

{\verbatim
\begin{center}
%code to be explained later
\begin{picture}(xext,yext)
\putshape1(x1,y1)[shape_parameters1]
|dots
\putshape(xn,yn)[shape_parametersn]
\end{picture}
\end{center}
|egroup

Here is an example of the first kind of macro.  I will have to explain a
few details before giving examples of the second.  The code
{\verbatim
\begin{center}
\resetparms
\square[A`B`C`D;f`g`h`k]
\end{center}
|egroup
produces the diagram
\begin{center}
\resetparms
\square[A`B`C`D;f`g`h`k]
\end{center}
and the code
{\verbatim
\begin{center}
\resetparms
\Atriangle[A`B`C;f`g`h]
\end{center}
|egroup
produces the diagram
\begin{center}
\resetparms
\Atriangle[A`B`C;f`g`h]
\end{center}
The reason for the shape name will be explained later.

If an arrow label extends especially high or low, the space
automatically expands to match.  For example,
{\verbatim
\begin{center}
\resetparms
\square[A`B`C`D;\sum_{i=1}^{\infty}`g`h`\Psi^A_k]
\end{center}
|egroup
will get the diagram
\begin{center}
\resetparms
\square[A`B`C`D;\sum_{i=1}^{\infty}`g`h`\Psi^A_k]
\end{center}
and so on.

Before going on, I will explain about the parameters.  Any shape will
have one or two parameters for height and/or length as well as one
parameter for each arrow in the diagram.  These are in addition to the
parameters used in the actual procedures.  For example, the procedure
{\verbatim
\begin{center}
\setsqparms[-1`0`2`-3;1000`700]
\square[A`B`C`D;f`g`h`k]
\end{center}
|egroup
gives the square:
\begin{center}
\setsqparms[-1`0`2`-3;1000`700]
\square[A`B`C`D;f`g`h`k]
\end{center}
The meaning is as follows.  The first four parameters refer to four
arrows in linguistic order (top, left, right, bottom).  A negative
number gives a backward arrow, while a zero causes it to be omitted.  A
parameter with an absolute value of 1 is ordinary, while 2 gives an
arrow with a tail (monomorphism) and 3 a double headed arrow
(epimorphism).  The last two parameters determine the width and height,
respectively, in units of 0.01 em (an em is the width of an M, the
widest letter in a font).

Although {\tt \\resetparms} works with any of the shapes (and simply
chooses
the default values of 1 for all arrows and 500 for height and width),
the remaining shapes use different names.  The various parameter setting
procedures are
{\verbatim
\setsqparms[#1`#2`#3`#4;#5`#6]
\settriparms[#1`#2`#3;#4]
\settripairparms[#1`#2`#3`#4`#5;#6]
\setrecparms[#1`#2]
|egroup
With the exception of the last, the parameters before the semicolon set
the arrow types and the one or two after set lengths.  In the last one,
both parameters are lengths.

Here are the shapes.  We have already seen the shape {\tt \\square}.
There are eight different triangles, all isoceles right triangles in
different orientations.  The names all have the form {\tt
\\xtriangle}, where x is the letter that most closely resembles the
actual shape of the triangle.

Here are the names, followed by a sample of each one:
\begin{list}{}{}
\item{\tt \\btriangle}\samepage
\begin{center}
\resetparms
\btriangle[A`B`C;f`g`h]
\end{center}
\pagebreak[0]
\item{\tt \\dtriangle}\samepage
\begin{center}
\resetparms
\dtriangle[A`B`C;f`g`h]
\end{center}
\pagebreak[0]
\item{\tt \\ptriangle}\samepage
\begin{center}
\resetparms
\ptriangle[A`B`C;f`g`h]
\end{center}
\pagebreak[0]
\item{\tt \\qtriangle}\samepage
\begin{center}
\resetparms
\qtriangle[A`B`C;f`g`h]
\end{center}
\pagebreak[0]
\item{\tt \\Atriangle}\samepage
\begin{center}
\resetparms
\Atriangle[A`B`C;f`g`h]
\end{center}
\pagebreak[0]
\item{\tt \\Vtriangle}\samepage
\begin{center}
\resetparms
\Vtriangle[A`B`C;f`g`h]
\end{center}
\pagebreak[0]
\item{\tt \\Ctriangle}\samepage
\begin{center}
\resetparms
\Ctriangle[A`B`C;f`g`h]
\end{center}
\pagebreak[0]
\item{\tt \\Dtriangle}\samepage
\begin{center}
\resetparms
\Dtriangle[A`B`C;f`g`h]
\end{center}
\end{list}

In addition, there are two special diagrams that come up often enough to
be worth having a special macros for.
\begin{list}{}{}
\item{\tt \\Atrianglepair}\samepage
\begin{center}
\resetparms
\Atrianglepair[A`B`C`D;f`g`h`k`l]
\end{center}
\item{\tt \\Vtrianglepair}\samepage
\begin{center}
\resetparms
\Vtrianglepair[A`B`C`D;f`g`h`k`l]
\end{center}
\end{list}
Finally, there is one special shape that is probably not used by very
many mathematicians.  Still I have it and the user might as well share
it.

{\verbatim
\recurse[A`B`R`C;s`f_0`f`t_0`t]
|egroup
\begin{center}
\resetparms
\recurse[A`B`R`C;s`f_0`f`t_0`t]
\end{center}

Notice what happens if the first parameter is empty:
{\verbatim
\recurse[`B`R`C;s`f_0`f`t_0`t]
|egroup
\begin{center}
\resetparms
\recurse[`B`R`C;s`f_0`f`t_0`t]
\end{center}

This takes care of the simple procedures.  The remaining procedures are
there as building blocks for more complicated diagrams.

These take the form
\vskip0pt\noindent
{\tt \\putshape(xpos,ypos)[shape\_parameters]}
\vskip0pt\noindent
where
shape is one of the 11 shapes (excluding {\tt \\recurse}) described
above
and the parms are as described there.  The parameters {\tt  xpos} and
{\tt  ypos} are the offsets from the lower left corner of the picture
measured in umits of .01 em in the usual \LaTeX\ fashion.

There is one
more shape which is simply an arrow with an attached label.  This is
called {\tt \\putmorphism} and it is used with the syntax
\vskip0pt\noindent
{\tt
\\putmorphism(xpos,ypos)(run,rise)[node1`node2`label]\{dist\}\{type\}\{loc\}}
\vskip0pt\noindent
Here the parameters {\tt  xpos} and
{\tt  ypos} are as above.  The slope is {\tt rise/run} where {\tt rise}
and {\tt run} are two numbers that give the slope in accordance with the
\LaTeX\ rules.  That is rise and run must have no common divisor and
must not exceed 4 in absolute value.  In addition, these procedures are
defined so that rise must not be positive and if it is zero, then run
must be positive.  In other words, all arrows must go in the linguistic
direction, downwards or to the right.  Arrows can be made to go in the
reverse direction as explained below.

The next three parameters are the node that appears first (in linguistic
order), the one that appears second and the arrow label.  The next
paramater is the distance in the sense of \LaTeX\ between the centers of
the nodes.  This means it is the horizontal component of the distance,
unless that is negative, in which case it is the vertical distance.  The
second parameter is the code for the arrowtype, an integer between -3
and 3 with the same meaning as explained above.  Using negative values
of this parameter allows one to draw an arrow that goes upwards or left.
Note, however, that node1 is always the node that is above or to the
left of node2.  The last parameter should have the value a,b,l,r or m.
The values a and b are used only for horizontal arrows and direct the
arrow label to be placed above or below the arrow.  The values l and r
are used for all other arrows and direct the label to be left or right
of the arrow.  Finally, a vertical arrow only can be given the parameter
m, in which case the arrow will be gapped and the label placed in the
middle of the gap.

In general, the simple macros are designed to be used as indicated
either in the {\tt \\begin\{center\} \ldots \\end\{center\}} environment
or in one of the others, usually
{\verbatim
\begin{equation}\begin{array}{c}
|dots
\end{array}\end{equation}|egroup

The \\putshape macros must be used inside a picture environment that is
normally placed inside a centering or similar environment.  The
reference point for the positioning parameters is determined as the
lower left corner of the smallest rectangle with sides parallel to the
coordinate axes that includes the center points of all nodes on its
border.  This rectangle will be degenerate in the case of a horizontal
or vertical morphism.  The reference point may either be on or outside
the actual figure.  For example, in the case of the btriangle, it is the
center of the lower left node, while for a qtriangle it is outside the
triangle itself being the fourth corner of the enclosing square.

The macros are made so that they fit together well.  That is why all
distances are from node centers to node centers.  If a vertex is part of
two shapes, it is probably best not to repeat it, since it is
conceivable that round-off errors will cause its two appearances to be
slightly offset.  It can either be omitted or, since the width is used
to determine the amount to shorten horizontal arrows, replaced by {\tt
\\phantom} versions.  Both methods are illustrated in the example given
at the end.

One more thing has to be described.  when beginning picture mode, you
have to allocate both a horizontal and vertical space.  the vertical
space is crucial and, while the horizontal is not as critical, it still
determines whether the figure is properly centered.  This partially
automated as follows.  You begin by giving values to two variables,
called {\tt \\xext} and {\tt\\yext} (for x-extent and y-extent, resp.).
This is done by the declarations
{\tt\\xext=xparm, \\yext=yparm}, where {\tt xparm} and {\tt yparm} are
the sum of all the horizontal and vertical distances used in the
procedures.  Of course, if two shapes are put side by side, the
x-extents will be the sum plus the space between them if any, while the
y-extent will be that of the larger one.  In figuring out these extents,
ignore the heights and widths of the vertices and arrow labels.  So what
these really measure is the distance between the two most distantly
separated arrows in the diagram, or rather it would if none was omitted.

These must be adjusted for the space occupied by the labels and nodes.
There are procedures to do this automatically, but they require an
assist from the user.  First, there are six raw procedures that can
carry out these adjustments.  They are {\verbatim \topadjust,
\botadjust, \leftadjust, \rightadjust, \leftsladjust, \rightsladjust
|egroup that take 3 parameters each and adjust {\tt\\xext} and
{\tt\\yext} for the heights and widths of the nodes.  They all take as
parameters two nodes and one arrow and figure out which one stands the
highest or sticks out the furthest and makes the appropriate adjustment.
The first four can be given the nodes in either order and the arrow
label comes third.  In the case of the last two, making adjustment for
slnat arrows of a triangle, the first parameter is the node that sticks
out the further, the second is the other node (which is the rarely the
determining one) and the third is the label.  If it is clear that a node
or label cannot possibly be the salient one, then it may safely be
omitted.

In practice, it is hardly ever necessary to use these procedures.  There
is a single procedure called {\tt\\adjust} that is used as follows:
{\verbatim
\adjust[tnode`tlabel;lnode`llabel;rnode`rlabel;bnode`blabel]
|egroup
It is usually necessary to specify only four parameters leaving the
remainder blank.  For example, if the top label is empty, then some node
(if there is more than one, always use the tallest; if in doubt use them
both concatenated).  If the label is not blank, it will almost stick up
further than the node.  If in doubt, give them both, but this should
rarely happen.  Similar remarks apply to the other pairs of parameters.

These adjustment procedures adjust not only {\tt\\xext} and {\tt\\yext}
but another pair of variables called {\tt\\xpos} and {\tt\\ypos}.  When
this has all been done, you can now begin picture mode with
{\verbatim
\begin{picture}(\xext,\yext)(\xoff,\yoff)
|dots
\end{picture}
|egroup

A somewhat baroque (but taken from an actual text) example illustrates
most of these points.
{\verbatim
\begin{center}
\xext=2100 \yext=2100
\adjust[`\mu;`T\eta'T;`\sigma;`T'T\eta']
\begin{picture}(\xext,\yext)(\xoff,\yoff)
\putmorphism(0,2100)(0,-1)[``T\eta'T]{1400}1l
\putmorphism(0,2100)(1,0)[TT`T`\mu]{700}1a
\putmorphism(0,2100)(1,-1)[`TTT'`TT\eta']{700}1l
\putmorphism(700,2100)(1,-1)[`TT'`T\eta]{700}1r
\put(700,1750){\makebox(0,0){1}}
\putmorphism(700,1420)(1,0)[\phantom{TTT'}`\phantom{TT'}`\mu
  T']{700}1a
\putmorphism(700,1380)(1,0)[\phantom{TTT'}`%
  \phantom{TT'}`T\sigma]{700}1b
\setsqparms[0`1`1`1;700`700]
\putsquare(700,700)[TTT'`TT'`TT'TT'`TT'T';`T\eta'TT'``]
\putmorphism(700,700)(1,0)[\phantom{TT'TT'}`%
  \phantom{TT'T'}`TT'\sigma]{700}1a
\put(300,1400){\makebox(0,0){2}}
\put(950,1050){\makebox(0,0){3}}
\settriparms[0`1`0;700]
\putbtriangle(1400,700)[``TT';T\eta'T'`id`]
\putmorphism(1400,700)(1,0)[\phantom{TT'T'}`%
  \phantom{TT'}`T\mu']{700}1a
\put(1600,1050){\makebox(0,0){6}}
\setsqparms[1`1`0`1;700`700]
\putsquare(0,0)[TT'T`\phantom{TT'TT'}`T'T`T'TT';%
  TT'T\eta'`\sigma T``T'T\eta']
\putmorphism(700,0)(1,0)[\phantom{T'TT'}`%
  \phantom{T'T'}`T'\sigma]{700}1b
\setsqparms[0`0`1`1;700`700]
\putsquare(1400,0)[``T'T'`T';``\sigma`\mu']
\putmorphism(700,700)(0,-1)[``\sigma TT']{700}1m
\putmorphism(1400,700)(0,-1)[``\sigma T']{700}1m
\put(300,350){\makebox(0,0){4}}
\put(1050,350){\makebox(0,0){5}}
\put(1750,350){\makebox(0,0){7}}
\end{picture}
\end{center}
|egroup
which produces
\begin{center}
\xext=2100 \yext=2100
\adjust[`\mu;`T\eta'T;`\sigma;`T'T\eta']
\begin{picture}(\xext,\yext)(\xoff,\yoff)
\putmorphism(0,2100)(0,-1)[``T\eta'T]{1400}1l
\putmorphism(0,2100)(1,0)[TT`T`\mu]{700}1a
\putmorphism(0,2100)(1,-1)[`TTT'`TT\eta']{700}1l
\putmorphism(700,2100)(1,-1)[`TT'`T\eta]{700}1r
\put(700,1750){\makebox(0,0){1}}
\putmorphism(700,1420)(1,0)[\phantom{TTT'}`\phantom{TT'}`\mu
  T']{700}1a
\putmorphism(700,1380)(1,0)[\phantom{TTT'}`%
  \phantom{TT'}`T\sigma]{700}1b
\setsqparms[0`1`1`1;700`700]
\putsquare(700,700)[TTT'`TT'`TT'TT'`TT'T';`T\eta'TT'``]
\putmorphism(700,700)(1,0)[\phantom{TT'TT'}`%
  \phantom{TT'T'}`TT'\sigma]{700}1a
\put(300,1400){\makebox(0,0){2}}
\put(950,1050){\makebox(0,0){3}}
\settriparms[0`1`0;700]
\putbtriangle(1400,700)[``TT';T\eta'T'`id`]
\putmorphism(1400,700)(1,0)[\phantom{TT'T'}`%
  \phantom{TT'}`T\mu']{700}1a
\put(1600,1050){\makebox(0,0){6}}
\setsqparms[1`1`0`1;700`700]
\putsquare(0,0)[TT'T`\phantom{TT'TT'}`T'T`T'TT';%
  TT'T\eta'`\sigma T``T'T\eta']
\putmorphism(700,0)(1,0)[\phantom{T'TT'}`%
  \phantom{T'T'}`T'\sigma]{700}1b
\setsqparms[0`0`1`1;700`700]
\putsquare(1400,0)[``T'T'`T';``\sigma`\mu']
\putmorphism(700,700)(0,-1)[``\sigma TT']{700}1m
\putmorphism(1400,700)(0,-1)[``\sigma T']{700}1m
\put(300,350){\makebox(0,0){4}}
\put(1050,350){\makebox(0,0){5}}
\put(1750,350){\makebox(0,0){7}}
\end{picture}
\end{center}

Here is a page of samples of the results from the various
{\tt\\putshapes}.  The code
{\verbatim
\begin{center}
\xext=3000 \yext=3500
\begin{picture}(\xext,\yext)(\xoff,\yoff)
\resetparms
\putsquare(0,0)[A`B`C`D;f`g`h`k]
\putbtriangle(0,1500)[A`B`C;f`g`h]
\putdtriangle(2200,1500)[A`B`C;f`g`h]
\putptriangle(0,3000)[A`B`C;f`g`h]
\putqtriangle(1000,500)[A`B`C;f`g`h]
\putCtriangle(1500,750)[A`B`C;f`g`h]
\putDtriangle(700,1000)[A`B`C;f`g`h]
\putAtriangle(2000,2200)[A`B`C;f`g`h]
\putAtrianglepair(0,2500)[A`B`C`D;f`g`h`k`l]
\putVtriangle(2000,3000)[A`B`C;f`g`h]
\putVtrianglepair(1000,2500)[A`B`C`D;f`g`h`k`l]
\end{picture}
\end{center}
|egroup
produces the diagram:
\begin{center}
\xext=3000 \yext=3500
\begin{picture}(\xext,\yext)(\xoff,\yoff)
\resetparms
\putsquare(0,0)[A`B`C`D;f`g`h`k]
\putbtriangle(0,1500)[A`B`C;f`g`h]
\putdtriangle(2200,1500)[A`B`C;f`g`h]
\putptriangle(0,3000)[A`B`C;f`g`h]
\putqtriangle(1000,500)[A`B`C;f`g`h]
\putCtriangle(1500,750)[A`B`C;f`g`h]
\putDtriangle(700,1000)[A`B`C;f`g`h]
\putAtriangle(2000,2200)[A`B`C;f`g`h]
\putAtrianglepair(0,2500)[A`B`C`D;f`g`h`k`l]
\putVtriangle(2000,3000)[A`B`C;f`g`h]
\putVtrianglepair(1000,2500)[A`B`C`D;f`g`h`k`l]
\end{picture}
\end{center}


Added material:

The catdoc file is out-of-date in two ways.  Most important is that all
the basic shapes (square and triangles) now accept optional arguments.
For example, the code
{\verbatim
\resetparms\square[A`B`C`D;f`g`h`k]
|egroup
is equivalent to simply
{\verbatim
|egroup
\square[A`B`C`D;f`g`h`k]
The code
{\verbatim
\settriparms[1`-1`2;750]\ptriangle[A`B`C;f`g`h]
|egroup
is equivalent to
{\verbatim
\ptriangle<1`-1`2;750>[A`B`C;f`g`h]
|egroup
and the code
{\verbatim
\settripairparms[1`-1`0`2`-3;1000]
\putAtrianglepair(500,750)[A`B`C`D;f`g`h`k`l]
|egroup
is equivalent to
{\verbatim
\putAtrianglepair<1`-1`0`2`-3;1000>(500,750)[A`B`C`D;f`g`h`k`l]
|egroup
Note that both the old and new syntax work.

The second new things are various shapes that I have found useful enough
to add over the last few months.  For example, I have both vertical and
horizontal coequalizers.  Had I had the need, I would have added
equalizers, and no doubt will at some future date.  Or maybe you will
and can send it to me.  These can all be found by reading the source
file, where they are sort of documented.
\end{document}
