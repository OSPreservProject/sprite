\def\UNX{UN$\star$X}
%\documentstyle[svma,draft]{report}
\documentstyle[svma]{report}
\begin{document}
\tableofcontents
\chapter{Springer-Verlag Document Styles for \LaTeX}
\chapterauthors{Ed Sznyter\footnote{Distributed Systems Group,
	Computer Science Department,
	Stanford University, Stanford, CA 94305--2140}}
\begin{abstract}
This document describes the \LaTeX\ styles ``svma,'' Springer-Verlag's
style package for multiauthored books,
and ``svsa,'' Springer-Verlag's style package for single-authored books.
These styles are variations of the standard ``report'' style, and
only the differences are noted here.
For a complete reference, see {\it \LaTeX: A Document Preparation System}
by Leslie Lamport.

A short example of a paper prepared with the ``svma'' style
may be found at the end of this document.
\end{abstract}

\section{Starting the Document}
\label{starting}

\subsection{Multi-Authored Books}
Each chapter in a multiauthored book may be written by different authors,
who write and test their parts as self-contained manuscripts.
The chapters are then submitted to the editor to be collected into the whole.

An author's document should start out as follows.
\begin{verbatim}
\documentstyle[svma,draft]{report}
\begin{document}
\chapter{Title of your Chapter}
\chapterauthors{Your Name\footnote{Your institution}\\
Another Author's Name\footnote{Their institution}}

\begin{abstract}
The abstract...
\end{abstract}
\end{verbatim}

and should end with
\begin{verbatim}
\end{document}
\end{verbatim}

All these are standard \LaTeX\ commands, except for \verb|\chapterauthors|.
\verb|\chapterauthors|
takes as an argument a list of authors, separated by the command \verb|\\|.
Each author should have a footnote which specifies their institution.

As a side-effect of this command,
the chapter name and the names of the authors will be placed in the
running heads automatically.

\subsection{Single-Authored Books}
A document that uses the ``svsa'' style should start as follows.
\begin{verbatim}
\documentstyle[svsa,draft]{report}
\begin{document}
\end{verbatim}
and each chapter starts with
\begin{verbatim}
\chapter{Chapter Title}
\end{verbatim}
Naturally, \verb|\chapterauthors| and the abstract is not necessary
for each chapter.  In this style, the chapter name and the section name
will be placed in the running heads.

\section{Sectioning Commands}
The following sectioning commands, as described in the \LaTeX\ book,
are available in this style:
\verb|\part|, \verb|\chapter|,
\verb|\section|, \verb|\subsection|, \verb|\subsubsection|,
\verb|\paragraph|, and \verb|\subparagraph|.
\verb|\part| is only used when the separate papers are put together
into one document.

By default, 
\verb|\part|, \verb|\chapter|, \verb|\section|, and \verb|\subsection|
are numbered and placed in the table of contents.
The commands
\begin{verbatim}
\setcounter{secnumdepth}{2}	% number section and subsections
\setcounter{tocdepth}{2}	% and put them in the table of contents
\end{verbatim}
found in the style file may be copied and changed by the editor if desired,
but should not be changed by the individual authors.

The important words in a section argument should be capitalized.

\section{Tables}
A special type of float, the \verb|stable|, has been added.  For example,

\begin{verbatim}
\begin{stable}{HYN Common Stock}{rcrp{2in}}
\multicolumn{1}{c}{Year}
	& \multicolumn{1}{c}{Price}
		& \multicolumn1c{Dividend}\\
\TableSubtitleRule
1971	& 41--54	& \$2.60\\
2	& 41--54	& 2.70	& This paragraph is long enough to span
				  at least two lines.
\TableFootnote{Prices are in dollars per share}
\end{stable}
\end{verbatim}
will produce table \ref{shorttable} here or at the top of the next page.

\begin{stable}{HYN Common Stock}{rcrp{2in}}
\multicolumn{1}{c}{Year}
	& \multicolumn{1}{c}{Price}
		& \multicolumn1c{Dividend}\\
\TableSubtitleRule
1971	& 41--54	& \$2.60\\
2	& 41--54	& 2.70	& This paragraph is long enough to span
				  at least two lines.
\TableFootnote{Prices are in dollars per share}
\label{shorttable}
\end{stable}

The first argument to \verb|stable| is the title of the table,
the second is the normal preamble that would be specified for a table
made with the \verb|tabular| environment.

The body of the \verb|stable| is specified in the same manner as
the body of \verb|tabular|.  The command \verb|\TableSubtitleRule|
will draw a horizontal rule below the column heads, with the proper
vertical spacing.  Note that you have to specify the centering of the
column heads, since \TeX\ doesn't know they're different from normal
column entries.

\verb|\TableFootnote| takes one argument, the text to go in the footnote
of the table.  It may appear anywhere in the body of the \verb|stable|.

\section{Acknowledgements}
Acknowledgements go at the end of a chapter, just before the bibliography.
\verb|\acknowledgements| is a sectioning command without any arguments,
and should be followed by a paragraph of text.

\section{The Bibliography}
\label{bibl}

Bibliographies are created with \verb|bibtex|.
Generally, you will have a central bibliography database,
which will have an entry for every paper you have ever referenced.

In the text of the document, put the command
\verb|\cite{CHERI83}| to reference the paper ``The Distributed V Kernel
and its Performance for Diskless Workstations'' by D.R. Cheriton.
Then, at the end of the document, but before the
\verb|\end{document}|, put
\begin{verbatim}
\bibliographystyle{alpha}
\bibliography{master}		% name of your bibliography database
\end{verbatim}

Then, run your document through \LaTeX.  It will complain
\begin{verbatim}
LaTeX Warning: Citation `CHERI83' on page 1 undefined.
\end{verbatim}
and
\begin{verbatim}
No file paper.bbl
\end{verbatim}

This will put some entries in the .aux file.  Now run \verb|bibtex paper|,
which will create the file paper.bbl, by extracting the proper
references from your bibliographic database.  \verb|bibtex| should not
give any error messages.
Then run \LaTeX\ again. Again, the warning
\begin{verbatim}
LaTeX Warning: Citation `CHERI83' on page 1 undefined.
\end{verbatim}
will appear.  Run \LaTeX\ yet again, and there should be no warnings.

This process must be repeated every time a new reference is added to
your document, so you will probably want to wait until the paper is
nearly finished before producing the bibliography.

\subsection{The Bibliography Database}
The bibliography entry for the preceding example would have been put
into the file master.bib thusly:
\begin{verbatim}
@Inproceedings(CHERI83, key ="CHERI83",
author="D.R.~Cheriton and W.~Zwaenepoel",
title="The Distributed V Kernel and its Performance for Diskless Workstations",
booktitle="Proceedings of the 9th Symposium on Operating System Principles",
Organization="ACM",
Year=1983)
\end{verbatim}
\subsection{What Bibtex Produces}

In the preceding example, the file paper.bbl might contain

\begin{verbatim}
\begin{thebibliography}{CZ83}

\bibitem[BL80]{lampson}
Ed. B.W.~Lampson.
\newblock {\it Distributed Systems: Architecture and Implementation}.
\newblock Springer-Verlag, 1980.

\bibitem[CZ83]{cheri83}
D.R.~Cheriton and W.~Zwaenepoel.
\newblock The distributed v kernel and its performance for diskless
  workstations.
\newblock In {\it Proceedings of the 9th Symposium on Operating System
  Principles}, ACM, 1983.

\end{thebibliography}
\end{verbatim}

If you don't have bibtex, this file could be produced by hand.  However,
bibtex should be part of the \LaTeX\ package.

\section{Producing the Collected Works in the ``svma'' Style}

The standard commands up to and including \verb|\begin{document}|,
and after and including
\verb|\end{document}| must be removed from each chapter.
Then, a master file should be produced, with the commands
\begin{verbatim}
\documentstyle[svma]{report}
\begin{document}
\tableofcontents
\end{verbatim}
followed by an \verb|\include| command for each chapter.
Front material may be in this master file, or a separate file that's included.
\verb|part| commands, if desired, should be put in this file, between the
\verb|include| commands.

\acknowledgements
Thanks to Gerhard Rossbach, Donna Moore, and Barbara Tompkins of
Springer-Verlag for their patience with my endless questions.

\subsection{Bibliography}
In order to produce a bibliography for each chapter in a collected work,
each chapter must be in a separate file, and be set up as in
section \ref{bibl} (except the chapter doesn't end in \verb|\end{document}|).
Run \LaTeX\ on the whole document, then bibtex on each chapter, then
re-run \LaTeX\ on the whole document twice.

A master bibliography for the entire work may be produced in the normal
manner, by putting the appropriate commands in the master file and
running bibtex on the master file.  You can have both bibliographies
for each chapter and for the entire work at the same time.

\appendix
\chapter{A Short Example}

This example will serve as a framework from which to start.

\begin{verbatim}
\documentstyle[svma,draft]{report}
\begin{document}
\chapter{Springer-Verlag Multi-Author Style for \LaTeX}
\chapterauthors{Ed Sznyter\footnote{Distributed Systems Group,
	Computer Science Department,
	Stanford University, Stanford, CA 94305--2140}}
\begin{abstract}
This document describes the style ``svma,'' Springer-Verlag's
style package for multiauthored books that are typeset using
\LaTeX.
\end{abstract}

\section{Starting the Document}
Each chapter in a multiauthored book may be written by a
different author, who write and test their parts as
self-contained manuscripts.

\subsection{The Bibliography Database}

\acknowledgements
Thanks go to the appropriate people.

\appendix
\chapter{A Short Example}

\bibliographystyle{alpha}
\bibliography{master}		% name of your bibliography database

\end{document}
\end{verbatim}

\chapter{Installing the Style Packages}
Optimally, the files svma.sty and svsa.sty
 should be placed in the same location
as the rest of the style files; on \UNX, that might be\linebreak[4]
/usr/local/lib/tex/macros/svma.sty.
If the user doesn't have the privileges to install system software,
the environment variable \verb|TEXINPUTS| may be set to look for
style files wherever desired.  For example, on \UNX, the csh command
\begin{verbatim}
setenv TEXINPUTS ".:$HOME/lib:/usr/local/lib/tex/macros"
\end{verbatim}
will cause \TeX\ to look in the current directory, the user's private
library directory, and the default system directory.  svma.sty will be
found if placed in any of these locations.  Equivalent commands are
available on most other operating systems.

Directly including the style file into your source, or using \verb|\input|
or \verb|\include| will not work, because the style files contain
special commands that are valid only when processed using
\verb|documentstyle|.
\end{document}
