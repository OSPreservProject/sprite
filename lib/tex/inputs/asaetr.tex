% asaetr.tex v0.9 01 Jan 91
% James Darrell McCauley (jdm5548@diamond.tamu.edu)
% This is part of a four file set:
%  asaetr.sty - LaTeX style for TRANSACTIONS OF THE ASAE (American Society for
%               Agricultural Engineers)
%  asaetr.bst - BiBTeX style for TRANSACTIONS OF THE ASAE
%  asaetr.tex - example usage of and documentation for above
%  asaetr.bib - part of the above example
%
\documentstyle{asaetr}
\tracingstats=1
\title{Utilizing \LaTeX\ and  B\kern-.05em{\large I}\kern-.025em{\large B}\kern-.08em\TeX\ for ASAE Papers}

\author{J.~D.~McCauley \associate } 
       % \and
       % E.~A.~Hiler, \fellow
       % \and
       % D.~C.~Bullock \student 
       % the format is: name \membership_grade, where membership_grade
       % is one of ( \member, \associate, \student, \affiliate, \fellow) 
\begin{document}
\bibliographystyle{asaetr}
\nocite{*}

\maketitle

\begin{abstract}
This document describes by example the use of {\tt asaetr.sty},
a \LaTeX\ style that somewhat conforms to the style used in
the {\em Transactions of the ASAE}\/.  \cite{asaeins} An accompanying \BiBTeX\
style file is also used to format the bibliography in a style
similar to that used by ASAE.
\keywords{\LaTeX,\ \BiBTeX,\ typesetting, ASAE Transactions.}
\end{abstract}

\levelone{Introduction}
\drop{T}he American Society of Agricultural Engineers editorial staff
encourages authors to submit electronic manuscripts in the following
formats: MacWrite, MS--Word, MS--Works (preferably Version 2.0),
WordPerfect (Version 5.0 or later), and WriteNow (ASAE, 1990). 
It's a pity they don't use \TeX\ or \LaTeX.  If they only knew
how good it could look \ldots

Well, the editorial staff
likes to do their own thing, but there's no reason why agricultural
engineers cannot try to conform to the style used by the {\em Transactions}
when writing Summer and Winter Meeting papers.
This file and the accompanying style files are my attempt to 
make the task of formatting papers easier for agricultural engineers.

\leveltwo{Objective}

My objective of this work was to develop \LaTeX\ and \BiBTeX\
style files for ASAE members.

\levelthree{Finer Objectives}
Well, that sounds pretty good, but I also wanted to
\begin{enumerate}
\item Make it easier on myself (`cause I'm a programmer, and programmers
are lazy),
\item Encourage my boss to switch to \TeX\, and
\item Show you the use of a  \verb# \levelthree # heading and the 
\verb# enumerate # environment.
\end{enumerate}

\levelone{Getting Started}

If you're unfamilar with \LaTeX, I would suggest picking up a copy
of the manual (Lamport, 1986).  \cite[note]{ll:86} If you're already familiar,
read on.

\renewcommand{\footnoterule}{} % no line
\begin{table}[hbp]
\footnotesize
\caption{Comparison of Publishing Tools}
\begin{center}
%\begin{minipage}{\columnwidth}
\renewcommand{\footnoterule}{} % no line
\begin{center}
\renewcommand{\thefootnote}{\fnsym{footnote}}

\begin{tabular}{crr} \thickhline  
Tool & \multicolumn{1}{c}{Learning Curve %\footnote{1.0 being easiest} 
}& \multicolumn{1}{c}{Support             %\footnote{10.0 being the best}
}\\ \thinhline 
FrameMaker & 5.0 & 6.0 \\  
Troff & 10.0 & 1.0 \\
\TeX                                     %\footnote{\TeX\ is the winner!}\/ 
& 7.0 & 10.0  \\ \thickhline
\end{tabular}
\linethickness{0pt}
\end{center}
%\end{minipage}
\end{center}
\end{table}

\leveltwo{The Preamble}
The preamble is where tell \LaTeX\ that you are going to use {\tt asaetr.sty}.
It's also where you list the authors and ASAE membership grades. Here's
an example:
\small \begin{verbatim}

\documentstyle{asaetr}
\title{Boring Title}
\author{U.\ B.\ Boring, \fellow \and 
       I.\ M.\ Young, \student \and
       R.\ U.\ Happy, \nonmember    }
\begin{document}
\maketitle

\end{verbatim} \normalsize
I have used up to four authors and still got it fit on one line.
Five authors may fit, depending on the lengths of the names. If
they don't all fit, two rows of authors will be formed. Membership
grades can be any of 
\small \begin{verbatim}

 \member, \associate, \student, 
 \affiliate, or \fellow. 

\end{verbatim} \normalsize
You can also use \verb#\nonmember#, but it has the same effect
as leaving the membership grade off. The \verb#\maketitle# command
simply tells \LaTeX\ to use this author and title information to
compose the title of the paper.

\leveltwo{The Abstract}

After the preamble comes the abstract. Here's an example:
\begin{verbatim}

\begin{abstract}
This is going to be short. See, I told you.
\keywords{brevity, terseness, words.}
\end{abstract}

\end{verbatim}
This should be straightforward enough.

\leveltwo{The Body}

The commands that you should be most familiar with to typeset the
body of your paper are the sectioning commands. They are
\begin{description}
\item[{\tt \\levelone:}] Same  level as the Introduction and Abstract.
\item[{\tt \\leveltwo:}] Secondary headings, such as ``Objectives'' above.
\item[{\tt \\levelthree:}] Third level headings for finer details.
\item[{\tt \\levelfour:}] Try to avoid fourth level headings.
\end{description}
The usage of these commands can be best described by an example:
\begin{verbatim}

 \leveltwo{The Body}

\end{verbatim}
This is the sectioning command for
the section you are now reading.  

Fourth level headings should be avoided because in the current
version of \verb# asaetr.sty# version 0.9, there is a font problem. 
Instead of using a slanted, small caps font, only a small caps 
font is used.  This will be worked out in later versions.

\leveltwo{Figures and Tables}

If you have a \PS\ printer available, it's highly recommended
that you use the \verb# \psfig# macros written by Trevor Darrell
to include high quality figures. Another useful utility for including
figures is {\tt fig} (or {\tt xfig} if you use X Windows).  Figure 1
was created in about 30 seconds using {\tt xfig}. You can get {\tt fig}
from {\tt cayuga.cs.rochester.edu} by anonymous ftp. Remember that
in {\em Transactions of the ASAE}, captions for figures go {\em below}
the figures.

\begin{figure}[htb]
  \setlength{\unitlength}{0.1mm} %{0.00625in}%{0.0125in}%
  \begin{center}
    \begin{picture}(181,181)(0,0)
      \thinlines \multiput(80,80)(-20,-20){4}{\framebox(80,80){}}
      \thicklines \put(0,0){\framebox(180,180){}} 
      \put(60,60){\line( 1, 1){ 60}}
    \end{picture}
  \end{center}
  \caption{Primitive figure.} 
\end{figure}

If plan to include tables, and if you want to have
footnotes within these tables, use the \verb#minipage# environment.
Contact your local \LaTeX\ guru or your {\em local guide} for
more information or see the source for this paper (comments within the
example table).
You'll notice in {\em Transactions of the ASAE} or in {\em Applied
Engineering in Agriculture} they use thicker lines for the top and
bottom rules in tables. Instead of having to change line thickness
yourself (as you do in using document style ``article'' and others),
you can use two macros that come with this style: \verb#\thickhline#
and \verb#\thinhline#.  See the example table in this document (Table 1).
Don't forget to put the caption {\em above} the table instead of below
it.

\leveltwo{The References}

This is where \BiBTeX\ comes into play.  The style file {\tt asaetr.bst}
is currently being developed. It comes close, but you may have
to edit some entries by hand. See Appendix B in (Lamport, 1986).

\levelone{Conclusion}

This file should serve as an excellent example of the use
of the style files.  If you still can't figure things out, hunt
up your local guru and ask him/her to explain \LaTeX\ and \BiBTeX\
style files.

%\bibliography{jdm5548,asaetr}
\bibliography{asaetr}
\end{document}








