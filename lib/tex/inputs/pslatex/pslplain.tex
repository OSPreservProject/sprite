% Changes to LPLAIN.TEX necessary to incorporate the PostScript
% fonts.  Modified by Mario Wolczko, March 1988 - Oct 1989
\input lplain

\typeout{PostScript LaTeX}
\everyjob{\typeout{PostScript-based LaTeX (\PSfontversion), Version 1.0}}

\makeatletter

\message{Math codes,}
% The maths codes for the various control sequences. The `"' means the
% number following is hexadecimal. The first number is the `class' (see
% page 154 of the TeXbook), the second is the family number (e.g., 2 =
% \textfont2---the symbol font), and the third and fourth specify the
% position of the character in the font.
%
% For example,
%    "22D7       % the code for \cdot
% is interpreted as follows:
%      2:  Class 2 (binary operation)
%      2:  Family 2 (\textfont2 is normally the Symbol font
%      D7: Position D7 (character D7 in hex, 215 in decimal, 327 in Octal
%  Note that the PostScript fonts have character positions greater than
%  127 (which is as high as they go in CM fonts).

% font families:
%	0 = roman
%	1 = italic (math and text)
%	2 = symbol
%	3 = math extension
%	\ms@ = slanted Symbol

\mathcode`\^^@="22D7 % \cdot
\mathcode`\^^A="32AF % \downarrow
\mathcode`\^^B="0\ms@61 % \alpha
\mathcode`\^^C="0\ms@62 % \beta
\mathcode`\^^D="22D9 % \land
\mathcode`\^^E="02D8 % \lnot
\mathcode`\^^F="32CE % \in
\mathcode`\^^G="0\ms@70 % \pi
\mathcode`\^^H="0\ms@6C % \lambda
\mathcode`\^^I="0\ms@67 % \gamma
\mathcode`\^^J="0\ms@64 % \delta
\mathcode`\^^K="32AD % \uparrow
\mathcode`\^^L="22B1 % \pm
\mathcode`\^^M="22C5 % \oplus
\mathcode`\^^N="02A5 % \infty
\mathcode`\^^O="02B6 % \partial
\mathcode`\^^P="32CC % \subset
\mathcode`\^^Q="32C9 % \supset
\mathcode`\^^R="22C7 % \cap
\mathcode`\^^S="22C8 % \cup
\mathcode`\^^T="0222 % \forall
\mathcode`\^^U="0224 % \exists
\mathcode`\^^V="22C4 % \otimes
\mathcode`\^^W="32AB % \leftrightarrow
\mathcode`\^^X="32AC % \leftarrow
\mathcode`\^^Y="32AE % \rightarrow
\mathcode`\^^Z="22B9 % \ne
\if@usecmsy
 \mathcode`\^^[="2\cmsy@05 % \diamond
\else
 \mathcode`\^^[="22E0 % \diamond (bit big)
\fi
\mathcode`\^^\="32A3 % \le
\mathcode`\^^]="32B3 % \ge
\mathcode`\^^^="32BA % \equiv
\mathcode`\^^_="22DA % \lor

\mathcode`\ ="8000 % \space
\mathcode`\!="5021
\mathcode`\'="02A2 % not active...already a superscript
\mathcode`\(="4228
\mathcode`\)="5229
\mathcode`\*="222A % \ast
\mathcode`\+="222B
\mathcode`\,="612C
\mathcode`\-="222D
\mathcode`\.="012E
\mathcode`\/="002F
\mathcode`\:="303A
\mathcode`\;="603B
\mathcode`\<="323C
\mathcode`\=="303D
\mathcode`\>="323E
\mathcode`\?="503F
\mathcode`\[="425B
\mathcode`\\="005C % \backslash
\mathcode`\]="525D
\mathcode`\_="8000 % \_
\mathcode`\{="427B
\mathcode`\|="027C
\mathcode`\}="527D
\mathcode`\^^?="12F2 % \smallint

% this has much better spacing than the default "f"
\mathcode`\f="02A6

\message{other codes,}
% Delimiter codes. The first (hex) number is the font family. The next 2
% numbers specify the start character within that family. The next number
% (4th) is the font family to go to if you need a bigger delimiter. The
% final 2 numbers specify the character within that font. For example,
%       "028300
% The \( (left paranthesis) starts with font 0 (normally \rm---Times Roman)
% and picks character "28 (`('). If a larger ( is needed then move to
% font family 3 (CM extension) and pick character 00 (0).
%
% Note: The TFM files should contain information about growth of a
% character within a font (i.e., without needing to go to another font). In
% the example, above, if character 0 was still not big enough then the TFM
% file would instruct character "10 to be used. If this was still not big
% enough then "20, followed by "40. If a larger parenthesis was required
% then character "60 would be used which is the top part of a left
% parenthesis. 

\delcode`\<="2E130A
\delcode`\>="2F130B
\delcode`\/="0A430E
\delcode`\|="27C30C
\delcode`\\="05C30F

\message{accents and new chars,}
% \chardef is used for commands in LR or paragraph mode. The character chosen
% specified by a hex number is that of the current font.
%   For example,
%      \chardef\ss="FB
% defines \ss to pick up character "FB (decimal 251) of the current font.
% (e.g., {\it\ss} would pick up an italic character.

\chardef\ss="FB
\chardef\ae="F1
\chardef\oe="FA
\chardef\o="F9
\chardef\AE="E1
\chardef\OE="EA
\chardef\O="E9
\chardef\i="F5 % \chardef\j="11 % no dotless j
% No dotless j so write out a message if \j is used.
\def\j{\@PSsub\j{j}}
\def\aa{\accent"CAa }
\chardef\l="F8
\chardef\L="E8

\def\_{\char'137 }
\def\AA{\leavevmode\setbox0\hbox{h}\dimen@\ht0\advance\dimen@-1ex%
  \rlap{\kern.2em\raise.67\dimen@\hbox{\char'312}}A}

% The following characters do not appear in PLAIN TeX
\chardef\cent="A2
\chardef\yen="A5
\chardef\currency="A8
\chardef\quotesingle="A9
\chardef\guillemotleft="AB
\chardef\guilsinglleft="AC
\chardef\guilsinglright="AD
\chardef\quotesinglbase="B8
\chardef\quotedblbase="B9
\chardef\guillemotright="BB
\chardef\ellipsis="BC

% take advantage of the ellipsis char
\def\@ldots{\mathinner\mathchar"00BC }
\def\ldots{\protect\pldots}
\def\pldots{\relax\ifmmode\@ldots\else\mbox{$\@ldots$}\fi}

\chardef\perthousand="BD
\chardef\ordfeminine="E3
\chardef\ordmasculine="EB

% Some more TeX characters
\def\dag{\mathhexbox0B2}
\def\ddag{\mathhexbox0B3}
\def\S{\mathhexbox0A7}
\def\P{\mathhexbox0B6}

\def\pb#1{\oalign{#1\crcr\hidewidth
    \vbox to.2ex{\hbox{\char'305}\vss}\hidewidth}}
\def\b{\protect\pb}

\def\pc#1{\setbox\z@\hbox{#1}\ifdim\ht\z@=1ex\accent"CB #1%
  \else{\ooalign{\hidewidth\char'313\hidewidth\crcr\unhbox\z@}}\fi}
\def\c{\protect\pc}

% The \og (ogonek) accent is not available in `standard' TeX
\def\pog#1{\setbox\z@\hbox{#1}\ifdim\ht\z@=1ex\accent"CE #1%
  \else{\ooalign{\hidewidth\char'316\hidewidth\crcr\unhbox\z@}}\fi}
\def\og{\protect\pog}

% there are two copyright symbols in PostScript, with and without serifs.
% The \@dosfchars command in lfonts.tex switches between them
% appropriately. 
\def\@copyrightTR{\mathhexbox2D3}
\def\@registeredTR{\mathhexbox2D2}
\def\@trademarkTR{\mathhexbox2D4}
\def\@copyrightH{\mathhexbox2E3}
\def\@registeredH{\mathhexbox2E2}
\def\@trademarkH{\mathhexbox2E4}

\let\pcopyright\@copyrightTR
\let\pregistered\@registeredTR
\let\ptrademark\@trademarkTR

% Definition of accents.
\def\`#1{{\accent"C1 #1}}
\def\'#1{{\accent"C2 #1}}
\def\v#1{{\accent"CF #1}} \let\^^_=\v
\def\u#1{{\accent"C6 #1}} \let\^^S=\u
\def\=#1{{\accent"C5 #1}}
\def\^#1{{\accent"C3 #1}} \let\^^D=\^
\def\.#1{{\accent"C7 #1}}
\def\H#1{{\accent"CD #1}}
\def\~#1{{\accent"C4 #1}}
\def\"#1{{\accent"C8 #1}}
% accents used in tabbing environments
\expandafter \let \csname a`\endcsname = \`
\expandafter \let \csname a'\endcsname = \'
\expandafter \let \csname a=\endcsname = \=

\let\@acci=\'
\let\@accii=\`
\let\@acciii=\=

\if@usecmmi
 \def\pt#1{{\edef\next{\the\font}\the\textfont\cmmifam\accent"7F\next#1}}
\else
 \def\t#1{\@PSnofont\t}
\fi

\def\*{\discretionary{\thinspace\the\textfont2\char'264}{}{}}

\message{math chars,}
% Maths character definitions.  The interpretation of the code is the same
% as for \mathcode (above).
\mathchardef\alpha="0\ms@61
\mathchardef\beta="0\ms@62
\mathchardef\gamma="0\ms@67
\mathchardef\delta="0\ms@64
\mathchardef\epsilon="0\ms@65
\mathchardef\zeta="0\ms@7A
\mathchardef\eta="0\ms@68
\mathchardef\theta="0\ms@71
\mathchardef\iota="0\ms@69
\mathchardef\kappa="0\ms@6B
\mathchardef\lambda="0\ms@6C
\mathchardef\mu="0\ms@6D
\mathchardef\nu="0\ms@6E
\mathchardef\xi="0\ms@78
\mathchardef\pi="0\ms@70
\mathchardef\rho="0\ms@72
\mathchardef\sigma="0\ms@73
\mathchardef\tau="0\ms@74
\mathchardef\upsilon="0\ms@75
\mathchardef\phi="0\ms@66
\mathchardef\chi="0\ms@63
\mathchardef\psi="0\ms@79
\mathchardef\omega="0\ms@77
\def\varepsilon{\@PSsub\varepsilon\epsilon}
\mathchardef\vartheta="0\ms@4A
\mathchardef\varpi="0\ms@76
\def\varrho{\@PSsub\varrho\rho}
\mathchardef\varsigma="0\ms@56
\mathchardef\varphi="0\ms@6A
\mathchardef\Gamma="0247
\mathchardef\Delta="0244
\mathchardef\Theta="0251
\mathchardef\Lambda="024C
\mathchardef\Xi="0258
\mathchardef\Pi="0250
\mathchardef\Sigma="0253
\mathchardef\Upsilon="02A1
\mathchardef\Phi="0246
\mathchardef\Psi="0259
\mathchardef\Omega="0257

\mathchardef\aleph="02C0
\def\hbar{{\mathchar"00C5\mkern-8muh}}
\mathchardef\imath="01F5
% \mathchardef\jmath="017C
\def\jmath{\@PSsub\jmath{j}}
\if@usecmmi
 \mathchardef\ell="0\cmmi@60
\else
 \def\ell{\@PSsub\ell{l}}
\fi
\mathchardef\wp="02C3
\mathchardef\Re="02C2
\mathchardef\Im="02C1
\mathchardef\partial="02B6
\mathchardef\infty="02A5
\mathchardef\@prime="02A2
\def\prime{\@prime\@warning{\string\prime\space produces a small prime in PS-LaTeX}}
\mathchardef\emptyset="02C6
\mathchardef\nabla="02D1
\def\surd{{\mathchar"12D6}}
\if@usecmsy
 \mathchardef\top="0\cmsy@3E
\else
 \def\top{\@PSnofont\top}
\fi
\mathchardef\bot="025E
\mathchardef\angle="02D0
\if@usecmsy
 \mathchardef\triangle="0\cmsy@34
\else
 \def\triangle{\@PSnofont\triangle}
\fi
\mathchardef\forall="0222
\mathchardef\exists="0224
\mathchardef\neg="02D8 \let\lnot=\neg
\if@usecmmi
 \mathchardef\flat="0\cmmi@5B
 \mathchardef\natural="0\cmmi@5C
 \mathchardef\sharp="0\cmmi@5D
\else
 \def\flat{\@PSnofont\flat}
 \def\natural{\@PSnofont\natural}
 \def\sharp{\@PSnofont\sharp}
\fi
\mathchardef\clubsuit="02A7
\mathchardef\diamondsuit="02E0
\mathchardef\fulldiamondsuit="02A8 % filled in black
\if@usecmsy
 \mathchardef\heartsuit="0\cmsy@7E
 \mathchardef\fullheartsuit="02A9 % filled in black
\else
 \mathchardef\heartsuit="02A9 % a reasonable approximation
\fi
\mathchardef\spadesuit="02AA

\mathchardef\smallint="12F2

\if@usecmmi
 \mathchardef\triangleleft="2\cmmi@2F
 \mathchardef\triangleright="2\cmmi@2E
\else
 \def\triangleleft{\@PSnofont\triangleleft}
 \def\triangleright{\@PSnofont\triangleright}
\fi
\if@usecmsy
 \mathchardef\bigtriangleup="2\cmsy@34
 \mathchardef\bigtriangledown="2\cmsy@35
\else
 \def\bigtriangleup{\@PSnofont\bigtriangleup}
 \def\bigtriangledown{\@PSnofont\bigtriangledown}
\fi
\mathchardef\wedge="22D9 \let\land=\wedge
\mathchardef\vee="22DA \let\lor=\vee
\mathchardef\cap="22C7
\mathchardef\cup="22C8
\mathchardef\ddagger="20B3
\mathchardef\dagger="20B2
\if@usecmsy
 \mathchardef\sqcap="2\cmsy@75
 \mathchardef\sqcup="2\cmsy@74
 \mathchardef\uplus="2\cmsy@5D
 \mathchardef\amalg="2\cmsy@71
 \mathchardef\diamond="2\cmsy@05
\else
 \def\sqcap{\@PSnofont\sqcap}
 \def\sqcup{\@PSnofont\sqcup}
 \def\uplus{\@PSnofont\uplus
 \def\amalg{\@PSnofont\amalg}
 \mathchardef\diamond="22E0 % close enough
\fi
\mathchardef\bullet="22B7
\if@usecmsy
 \mathchardef\wr="2\cmsy@6F
\else
 \def\wr{\@PSnofont\wr}
\fi
\mathchardef\div="22B8
\if@usecmsy
 \mathchardef\odot="2\cmsy@0C
 \mathchardef\oslash="2\cmsy@0B
\else
 \def\odot{\@PSnofont\odot}
 \def\oslash{\@PSnofont\oslash}
\fi
\mathchardef\otimes="22C4
\if@usecmsy
 \mathchardef\ominus="2\cmsy@09
\else
 \def\ominus{\@PSnofont\ominus}
\fi
\mathchardef\oplus="22C5
\if@usecmsy
 \mathchardef\mp="2\cmsy@07
 \mathchardef\circ="2\cmsy@0E
 \mathchardef\bigcirc="2\cmsy@0D
\else
 \def\mp{\@PSnofont\mp}
 \def\circ{\@PSnofont\circ}
 \def\bigcirc{\@PSnofont\bigcirc}
\fi
\mathchardef\pm="22B1
\mathchardef\setminus="205C % for set difference A\setminus B
\mathchardef\cdot="22D7
\mathchardef\ast="222A
\mathchardef\times="22B4
\if@usecmmi
 \mathchardef\star="2\cmmi@3F
\else
 \def\star{\@PSnofont\star}
\fi

\mathchardef\propto="32B5
\mathchardef\mid="327C
\if@usecmsy
 \mathchardef\sqsubseteq="3\cmsy@76
 \mathchardef\sqsupseteq="3\cmsy@77
 \mathchardef\parallel="3\cmsy@6B
 \mathchardef\dashv="3\cmsy@61
 \mathchardef\vdash="3\cmsy@60
 \mathchardef\nearrow="3\cmsy@25
 \mathchardef\searrow="3\cmsy@26
 \mathchardef\nwarrow="3\cmsy@2D
 \mathchardef\swarrow="3\cmsy@2E
\else
 \def\sqsubseteq{\@PSnofont\sqsubseteq}
 \def\sqsupseteq{\@PSnofont\sqsupseteq}
 \def\parallel{\@PSnofont\parallel}
 \def\dashv{\@PSnofont\dashv}
 \def\vdash{\@PSnofont\vdash}
 \def\nearrow{\@PSnofont\nearrow}
 \def\searrow{\@PSnofont\searrow}
 \def\nwarrow{\@PSnofont\nwarrow}
 \def\swarrow{\@PSnofont\swarrow}
\fi
\mathchardef\Leftrightarrow="32DB
\mathchardef\Leftarrow="32DC
\mathchardef\Rightarrow="32DE
\if@usecmsy
 \mathchardef\cmsyLeftrightarrow="3\cmsy@2C
 \mathchardef\cmsyLeftarrow="3\cmsy@28
 \mathchardef\cmsyRightarrow="3\cmsy@29
\fi
\mathchardef\ne="22B9 \let\neq=\ne
\def\not#1{\rlap{\kern.5em\lower.25ex\hbox{\mathhexbox2A4}}#1}
\mathchardef\leq="32A3 \let\le=\leq
\mathchardef\geq="32B3 \let\ge=\geq
\if@usecmsy
 \mathchardef\succ="3\cmsy@1F
 \mathchardef\prec="3\cmsy@1E
\else
 \def\succ{\@PSnofont\succ}
 \def\prec{\@PSnofont\prec}
\fi
\mathchardef\approx="32BB
\if@usecmsy
 \mathchardef\succeq="3\cmsy@17
 \mathchardef\preceq="3\cmsy@16
\else
 \def\succeq{\@PSnofont\succeq}
 \def\preceq{\@PSnofont\preceq}
\fi
\mathchardef\supset="32C9
\mathchardef\subset="32CC
\mathchardef\supseteq="32CA
\mathchardef\subseteq="32CD
\mathchardef\in="32CE
\mathchardef\notsub="32CB % this is a new one
\mathchardef\ni="3227 \let\owns=\ni
\if@usecmsy
 \mathchardef\gg="3\cmsy@1D
 \mathchardef\ll="3\cmsy@1C
\else
 \def\gg{\@PSnofont\gg}
 \def\ll{\@PSnofont\ll}
\fi
\mathchardef\leftrightarrow="32AB
\mathchardef\leftarrow="32AC \let\gets=\leftarrow
\mathchardef\rightarrow="32AE \let\to=\rightarrow
\if@usecmsy
 \mathchardef\cmsyleftrightarrow="3\cmsy@24
 \mathchardef\cmsyleftarrow="3\cmsy@20
 \mathchardef\cmsyrightarrow="3\cmsy@21
 \mathchardef\mapstochar="3\cmsy@37 \def\mapsto{\mapstochar\cmsyrightarrow}
 \mathchardef\simeq="3\cmsy@27
\else
 \def\mapstochar{\@PSnofont\mapstochar}
 \def\simeq{\@PSnofont\simeq}
\fi
\mathchardef\sim="327E
\mathchardef\perp="325E
\mathchardef\equiv="32BA
\if@usecmsy
 \mathchardef\asymp="3\cmsy@10
\else
 \def\asymp{\@PSnofont\asymp}
\fi
\if@usecmmi
 \mathchardef\smile="3\cmmi@5E
 \mathchardef\frown="3\cmmi@5F
 \mathchardef\leftharpoonup="3\cmmi@28
 \mathchardef\leftharpoondown="3\cmmi@29
 \mathchardef\rightharpoonup="3\cmmi@2A
 \mathchardef\rightharpoondown="3\cmmi@2B
\else
 \def\smile{\@PSnofont\smile}
 \def\frown{\@PSnofont\frown}
 \def\leftharpoonup{\@PSnofont\leftharpoonup}
 \def\rightharpoonup{\@PSnofont\rightharpoonup}
 \def\leftharpoondown{\@PSnofont\leftharpoondown}
 \def\rightharpoondown{\@PSnofont\rightharpoondown}
\fi

\if@usecmmi
 \mathchardef\cmminus="2\cmsy@00
 \def\relbar{\mathrel{\smash{\cmminus}}}
 \mathchardef\lhook="3\cmmi@2C
 \def\hookrightarrow{\lhook\joinrel\cmsyrightarrow}
 \mathchardef\rhook="3\cmmi@2D
 \def\hookleftarrow{\cmsyleftarrow\joinrel\rhook}
\else
 \def\lhook{\@PSnofont\lhook}
 \def\rhook{\@PSnofont\rhook}
\fi

% These are unavailable because we don't have cmr10 loaded (for the =
% sign), and wouldn't need it otherwise.  (I've tried simulating an
% equals sign with rules, but it looks awful, and doesn't scale--miw)
\def\Longrightarrow{\@PSsub\Longrightarrow\Rightarrow}
\def\Longleftarrow{\@PSsub\Longleftarrow\Leftarrow}

\if@usecmsy
 \def\longrightarrow{\protect\@lra}
   \def\@lra{\relbar\joinrel\cmsyrightarrow}
 \def\longleftarrow{\protect\@lla}
    \def\@lla{\cmsyleftarrow\joinrel\relbar}
 \def\longmapsto{\mapstochar\longrightarrow}
 \def\longleftrightarrow{\cmsyleftarrow\joinrel\cmsyrightarrow}
 \def\Longleftrightarrow{\cmsyLeftarrow\joinrel\cmsyRightarrow}
\fi

\def\rightarrowfill{$\m@th\cmminus\mkern-6mu%
  \cleaders\hbox{$\mkern-2mu\cmminus\mkern-2mu$}\hfill
  \mkern-6mu\mathord\cmsyrightarrow$}
\def\leftarrowfill{$\m@th\mathord\cmsyleftarrow\mkern-6mu%
  \cleaders\hbox{$\mkern-2mu\cmminus\mkern-2mu$}\hfill
  \mkern-6mu\cmminus$}

\mathchardef\cdotp="62D7 % cdot as a punctuation mark

% Some additional maths definitions available in the PostScript fonts
\mathchardef\therefore="025C
\mathchardef\cret="02BF
\mathchardef\seconds="02B2
\mathchardef\degree="02B0


% Upright Greek symbols.
\def\ualpha{\mathhexbox261}
\def\ubeta{\mathhexbox262}
\def\ugamma{\mathhexbox267}
\def\udelta{\mathhexbox264}
\def\uepsilon{\mathhexbox265}
\def\uzeta{\mathhexbox27A}
\def\ueta{\mathhexbox268}
\def\utheta{\mathhexbox271}
\def\uiota{\mathhexbox269}
\def\ukappa{\mathhexbox26B}
\def\ulambda{\mathhexbox26C}
\def\umu{\mathhexbox26D}
\def\unu{\mathhexbox26E}
\def\uxi{\mathhexbox278}
\def\upi{\mathhexbox270}
\def\urho{\mathhexbox272}
\def\usigma{\mathhexbox273}
\def\utau{\mathhexbox274}
\def\uupsilon{\mathhexbox275}
\def\uphi{\mathhexbox266}
\def\uchi{\mathhexbox263}
\def\upsi{\mathhexbox279}
\def\uomega{\mathhexbox277}
\def\uvartheta{\mathhexbox24A}
\def\uvarpi{\mathhexbox276}
\def\uvarsigma{\mathhexbox256}
\def\uvarphi{\mathhexbox26A}

% Maths accents
\def\acute{\mathaccent"70C2 }
\def\grave{\mathaccent"70C1 }
\def\ddot{\mathaccent"70C8 }
\def\tilde{\mathaccent"70C4 }
\def\bar{\mathaccent"70C5 }
\def\breve{\mathaccent"70C6 }
\def\check{\mathaccent"70CF }
\def\hat{\mathaccent"70C3 }
\if@usecmmi
 \def\vec{\mathaccent"0\cmmi@7E }
\else
 \def\vec{\@PSnofont\vec}
\fi
\def\dot{\mathaccent"70C7 }

\if@usecmsy
 \def\Vert{\delimiter"\cmsy@6B30D } \let\|=\Vert
 \def\vert{\delimiter"\cmsy@6A30C }
 \def\uparrow{\delimiter"3\cmsy@22378 }
 \def\downarrow{\delimiter"3\cmsy@23379 }
 \def\updownarrow{\delimiter"3\cmsy@6C33F }
 \def\Uparrow{\delimiter"3\cmsy@2A37E }
 \def\Downarrow{\delimiter"3\cmsy@2B37F }
 \def\Updownarrow{\delimiter"3\cmsy@6D377 }
 \def\backslash{\delimiter"\cmsy@6E30F } % for double coset G\backslash H
 \def\rangle{\delimiter"5\cmsy@6930B }
 \def\langle{\delimiter"4\cmsy@6830A }
 \def\rbrace{\delimiter"5\cmsy@67309 }
 \def\lbrace{\delimiter"4\cmsy@66308 }
 \def\rceil{\delimiter"5\cmsy@65307 }
 \def\lceil{\delimiter"4\cmsy@64306 }
 \def\rfloor{\delimiter"5\cmsy@63305 }
 \def\lfloor{\delimiter"4\cmsy@62304 }
\else	% these are not as good, and the small ones don't match the large
 \def\vert{\delimiter"27C30C }
 \def\uparrow{\delimiter"32AD378 }
 \def\downarrow{\delimiter"32AF379 }
 \def\updownarrow{\delimiter"333F33F }
 \def\Uparrow{\delimiter"32DD37E }
 \def\Downarrow{\delimiter"32DF37F }
 \def\Updownarrow{\delimiter"3377377 }
 \def\backslash{\delimiter"05C30F } % for double coset G\backslash H
 \def\rangle{\delimiter"52F130B }
 \def\langle{\delimiter"42E130A }
 \def\rbrace{\delimiter"527D309 } \let\}=\rbrace
 \def\lbrace{\delimiter"427B308 } \let\{=\lbrace
 \def\rceil{\delimiter"52F9307 }
 \def\lceil{\delimiter"42E9306 }
 \def\rfloor{\delimiter"52FB305 }
 \def\lfloor{\delimiter"42EB304 }
\fi

\if@usecmsy
 \def\sqrt{\radical"\cmsy@70370 }
\else
 \def\sqrt{\radical"370370 } % otherwise go straight to CMEX
\fi

\mathchardef\cong="3240
\mathchardef\notin="32CF

% LaTeX's \fnsymbol uses characters which are not available.
\def\@fnsymbol#1{\ifcase#1\or *\or \dagger\or \ddagger\or 
   \mathchar "0A7\or \mathchar "0B6\or \mathchar "0A8\or **\or \dagger\dagger
   \or \ddagger\ddagger \else\@ctrerr\fi\relax}

\makeatother

\typeout{Any more modifications...?}

% Absolve Leslie of all responsibility for any mistakes...
\def\fmtname{pslplain}\def\fmtversion{2.09-3/10/89} 
