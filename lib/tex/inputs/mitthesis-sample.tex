% mitthesis-sample.tex  12 Sep 89
% by Stephen Gildea <gildea@erl.mit.edu>
\documentstyle[11pt,mitthesis]{report}
\begin{document}

\title{Mixed Mode Oblique Ionograms in Transverse Shear Deformations
of Epitaxial Laminar Turbulent Circular Cylindrical Shells}

\author{J. Casey Salas}
\prevdegrees{B.S., University of California (1978) \\
  S.M., Massachusetts Institute of Technology (1981)}
\department{Department of Electrical Engineering and Computer Science}

% If the thesis is for two degrees simultaneously, list them both
% separated by \and like this:
% \degree{Doctor of Philosophy \and Master of Science}
\degree{Doctor of Philosophy}

\degreemonth{February}
\degreeyear{1987}
\thesisdate{December 10, 1986}

% If the thesis is copyright by the Institute, leave this line out and
% the standard copyright line will be used instead.
\copyrightnotice{J. Casey Salas, 1986}

% If there is more than one supervisor, use the \supervisor command
% once for each.
\supervisor{John D. Galli}{Director, Sound Instrument Laboratory}

% The \supervisor command takes an optional argument in case you
% want to label a person other than "Thesis Supervisor".  For example,
% \supervisor[Thesis Co-supervisor]{Michael Prange}{Senior Research Scientist}

% this is the department committee chairman, not the thesis committee chairman
\chairman{Arthur C. Smith}
  {Chairman, Departmental Committee on Graduate Students}

% Make the titlepage based on the above information.  If you need
% something special and can't use the standard form, you can specify
% the exact text of the titlepage yourself.  Put it in a titlepage
% environment and leave blank lines where you want vertical space.
% The spaces will be adjusted to fill the entire page.  The dotted
% lines for the signatures are made with the \signature command.
\maketitle

% The abstractpage environment sets up everything on the page except
% the text itself.  The title and other header material are put at the
% top of the page, and the supervisors are listed at the bottom.  A
% new page is begun both before and after.  Of course, an abstract may
% be more than one page itself.  If you need more control over the
% format of the page, you can use the abstract environment, which puts
% the word "Abstract" at the beginning and single spaces its text.

\begin{abstractpage}
Recursive lattice least squares (RLLS) was selected as the baseline
algorithm for the identification.  Simulation results on a
one-dimensional LSS demonstrated that it provided good estimates, was
not ill-conditioned in the presence of under-excited modes, allowed
activity by a supervisory control system which prevented damage to the
LSS or excessive drift, and was capable of real-time processing for
typical LSS models.

The editor generates data files
representing this model that can be used as input to a ray-tracing
program.  The description of the file format is
written in a high-level declarative language, so that the editor can
be easily modified to support other formats.
\end{abstractpage}

% The text of the thesis itself begins here.

% You may want to put a \tableofcontents command here.

% An acknowledgments section (probably begun with
% \section*{Acknowledgments}) might go here.

\chapter{Introduction}

\section{Background}
Observed elevation changes near the 1979 hypocenter 
(on the southern end of Imperial fault) and in the Brawley Seismic Zone show
significant deviations from those predicted by models of fault slip
inferred from strong ground motion measurements.  Specifically, the
geodetic data suggest that 
slip on the Imperial fault is significantly lower in the
vicinity of the earthquake hypocenter
than along the central and northern sections of the
fault.  In addition, 
there is marginal evidence that the dip of the Imperial fault
changes along strike from approximately vertical
just north of the US-Mexico border to between $70^\circ$ and $80^\circ$ near
the northernmost extent of the 1979 surface break. This change in dip
may be related to a change in local strike along the fault.  Large
elevation changes ($>15$ cm) also occur within the Brawley Seismic
Zone well north of the primary surface faulting. 

While these 
movements are consistent with a number of possible fault
models, our prefered interpretation based on geodetic and seismic
observations (aftershock locations and focal mechanisms) 
involves right lateral, aseismic slip on a northwest 
striking fault along the east side of the Brawley Seismic Zone and
conjugate left lateral faulting on a northeast striking fault
(possibly associated with a M=5.8 aftershock).\footnote{Buried creep on this 
same right lateral fault in the Brawley Seismic Zone can also 
account for vertical deformation during the postseismic period of the
1940, M=7.1 Imperial Valley earthquake as well as deformation during the
interseismic period between the 1940 and 1979 events.}

Substantial
right lateral slip in the Brawley Seismic Zone suggests that a
significant part of the shear strain released during and following the
1940 and 1979 earthquakes on the Imperial fault is transferred through
the Brawley Seismic Zone to the southern end of the San Andreas
fault.

Elevation changes in the Imperial Valley, California derived from
repeated leveling surveys for the time period including the 1979,
M=6.6 earthquake, provide some constraints on fault geometry and slip
distribution associated with this event.  Many of the first order
features of the observed vertical movements are well matched by simple
models consisting of variable slip on planar faults in an elastic
half-space using fault offsets
inferred from strong ground motion observations (Archuleta, 1984) and measured
after-slip.  The geodetic and seismic observations suggest that 
significant slip is confined to depths above
13 km with maximum right lateral offset reaching about 1.5 m on the
Imperial fault.  Dip slip occurs predominantly in
the sediments on the upper 5 km of the Imperial fault  and on the
Brawley fault.  Right lateral after-slip is confined to the upper 5 km
of the Imperial fault and reaches about 30 cm for the period 1979 to
1981.

\end{document}
