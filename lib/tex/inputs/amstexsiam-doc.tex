% This is the file siamdoc.tex.  Typeset with plain TeX.
\magnification=\magstephalf
\tolerance=1000
\def\beginverbatim{\par\begingroup\setupverbatim\doverbatim}
{\catcode`\|=0 \catcode`\\=12 % | is temporary escape character
  |obeylines|gdef|doverbatim^^M#1\endverbatim{#1|endgroup}}
\def\setupverbatim{\tentt \obeylines \uncatcodespecials \obeyspaces}
{\obeyspaces\global\let =\ } % let active space = control space
\def\uncatcodespecials{\def\do##1{\catcode`##1=12}\dospecials}
{\catcode`\^^M=13 \gdef\gobblecr{\ifnextchar
{\gobble}{\ignorespaces}}}
{\catcode`\ =\active\gdef\vobeyspaces{\catcode`\ \active \let \xobeysp}}
 \def\xobeysp{\leavevmode{} }
\begingroup \catcode `|=0 \catcode `[= 1
\catcode`]=2 \catcode `\{=12 \catcode `\}=12
\catcode`\\=12 |gdef|@xverbatim#1\end{verbatim}[#1|end[verbatim]]
|gdef|@sxverbatim#1\end{verbatim*}[#1|end[verbatim*]]
|endgroup
\def\makeother#1{\catcode`#112\relax}
\def\verb{\begingroup \tt \uncatcodespecials
\averb}
\def\sverb#1{\def\tempa ##1#1{##1\endgroup}\tempa}
\def\averb{\obeyspaces \frenchspacing \sverb}
\font\textfontii = cmsy10
\font\eightpt = cmr8
\def\heading#1{\medskip\noindent{\bf #1.\ }}
\def\AmSTeX{{\textfontii A}\kern-.1667em\lower.5ex\hbox
 {\textfontii M}\kern-.125em{\textfontii S}-\TeX}
\def\qed{\ifhmode\unskip\nobreak\fi\ifmmode\ifinner\else\hskip5pt\fi\fi
 \hbox{\hskip5 pt \lower 1.5 pt\hbox{\vrule width .2 pt 
 \vbox{\hrule width 4 pt height .2 pt \vskip 7.1 pt
 \hrule width 4 pt height .2 pt }\unskip\vrule width .2 pt}}}
\centerline{\bf USING  THE \AmSTeX\ SIAM STYLE FILE}
\medskip
{\eightpt\centerline{DOUGLAS N. ARNOLD and BRADLEY J. LUCIER}}
\bigskip
This document explains the use  of the \AmSTeX\ SIAM style file to produce
a paper that is well on its way to being
typographically acceptable for publication
in a SIAM journal.  The \AmSTeX\ SIAM style file (amstexsiam.sty) is
a modification of the \AmSTeX\ style file, amsppt.sty.
We assume familiarity with \AmSTeX\ and amsppt,
as documented in
{\it The Joy of \TeX\ }
by Michael Spivak.

Most of the points introduced below are illustrated in
the nonsense paper {\it A sample paper to illustrate the
\AmSTeX\ SIAM style file}.

\heading{Publication information}  When the paper has been accepted,
this must be noted in the \TeX\ input file using the macro \verb"\accepted",
and various information must be given
after the \verb"\documentstyle" command to correctly format
the first page of the paper and the running heads.  This information
consists of the beginning page number of the article,
the journal name, the issue volume, the issue number, 
the month of appearance, the year of appearance, 
the place of the paper in the issue, the head for even pages, and
the head for odd pages, as in the following example:
\medskip
\beginverbatim
\accepted
\pageno=10
\def\journalname{{\sixrm SIAM J. M{\fiverm ATH.} A{\fiverm NAL.}}}
\def\issuevolume{1}
\def\issuenumber{2}
\def\issuemonth{February}
\def\issueyear{1988}
\def\placenumber{002}
\def\evenhead{bradley j\. lucier and douglas n\. arnold}
\def\oddhead{a sample paper}.
\endverbatim
\medskip
\noindent The even and odd heads can be given even before the paper has been
accepted.

\heading{Title} Use \verb"\title" and \verb"\endtitle".
Type the title in all caps.
Attach any footnotes that apply to the paper
as a whole to the title.

\heading{Author} Use \verb"\author" and \verb"\endauthor".
Type the author's name in all caps.  For multiple authors type the word
``and'' in lowercase.

\heading{Author's affiliation}  Give the  affiliation of each author
in a footnote attached to each author's name.  Statements
acknowledging support should be contained in the same footnote.
Do {\it not} use the \verb"\affil" or \verb"\address" macros.

\heading{Footnotes} Use \verb"\footnote".
For the topmatter, use footnote symbols.
In the rest of the paper
use numbered footnotes. 
The order of footnote symbols is star (*\thinspace = \thinspace\verb"*"), dagger
(\dag\thinspace =\thinspace \verb"\dag"), double-dagger (\ddag\thinspace =\thinspace \verb"\ddag"),
section-marker (\S\thinspace = \thinspace\verb"\S").

\heading{Abstract} Use \verb"\abstract".

\heading{Keywords} Use \verb"\keywords" with commas between keywords
but without ending punctuation.

\heading{Subject classifications} Use \verb"\subjclass"
with commas between classifications
but without ending punctuation.

\heading{Headings} Use \verb"\subheading" without ending punctuation.
The argument shall begin with the section number followed by a period
and then the section name with only the first word capitalized.
Very long papers could use \verb"\heading" and \verb"\subheading".  If
\verb"\heading" is used start heading name with a section mark
(\S\thinspace =\thinspace\verb"\S") and the section number followed by a period and the
section title with the first letter of each major word capitalized.

\heading{Equation numbers}  Use \verb"\tag".

\heading{Theorems, etc} Use \verb"\proclaim" and \verb"\endproclaim"
for theorems, lemmas, corollaries, claims, propositions, etc.  The
argument to \verb"\proclaim" takes no ending punctuation.

\heading{Definitions} Use \verb"\proclaim" and \verb"\endproclaim",
but use \verb"\rm" to set the body of the definition in roman.

\heading{Proofs, etc} Use \verb"\demo" and \verb"\enddemo"
for proofs, examples, remarks, cases, etc.  The
argument to \verb"\demo" takes no ending punctuation.  The
end-of-proof mark is an open box (\qed\thinspace =\thinspace\verb"\qed").

\heading{Lists and sublists}  Use \verb"\roster" and \verb"\endroster".
Rosters use default labels of the form (1), (2), etc.
Subrosters are permitted.  Each subroster {\it must} be enclosed
in a separate pair of braces. Subrosters are not standard \AmSTeX.
 
\heading{References} These can be done exactly as in {\it The Joy
of \TeX.}  Use \verb"\Refs" and \verb"\ref".

\heading{Table of contents} There are no special macros provided to set
a table of contents.  If a table of contents is desired, follow these
guidelines.  Switch to eight point type (\verb"\eightpoint").
Center the word ``CONTENTS'' in caps.  Capitalize only the first word
of entries.  Runover lines align with the start of entries.
Place the section number flush left with one em to section title,
spaced dots from end of entry to page number, and two ems between the
dots and page number.  Leave 18--20 points of vertical space, baseline
to baseline, to the text above and below.

\heading{Figure legends} Use \verb"\topspace" or \verb"\midspace".
Set the caption as, for example,
\verb"\topspace{3 in}\caption{\eightpoint{\smc Fig\. 1.\enspace}\it Type caption here.}".

\heading{Tables} There are no special macros provided to set
tables.  Follow these guidelines.
Switch to eight point type (\verb"\eightpoint").
Center the table on the page width.  Put a minimum
of  18 points
space (baseline to baseline) above and/or below the table.
Center the label over the table.  The label
should be of the form \verb"{\smc Table~1.}"  The title of the table should
be in italics centered over the table with only the first word capitalized.
Column headings may be in set in six point type if necessary to fit properly
over the columns.  Place  half point rules
above and below the column headings, and at the end of the table.

\heading{New control sequence in the \AmSTeX\ SIAM style file}
The amount that each roster is indented is controlled by the value of
\verb"\rosterindent".  For most lists the default value need not
be changed.

\heading{Fonts} The font \verb"cmcsc10" is
used at 8 point (\verb"cmcsc10 at 8 pt") so there is a
caps-and-small-caps font for use in abstracts, etc.  You must
have the appropriate \verb"pxl", \verb"pk" or \verb"gf" file for
your printer driver. (See your local \TeX\ guru.  He or she may 
generate the font in the right size using Metafont.) 
Or, you can redefine the font \verb"\eightsmc"
to be whatever you can use for an eight point caps-and-small-caps font.
If you have absolutely no eight point caps-and-small-caps font, then
give the command \verb"\font\eightsmc=cmr8" to substitute
eight point roman.  SIAM can change it later if necessary.

The file uses
the newer \verb"cm" (computer modern) family of fonts (\verb"cmr9",
\verb"cmr8", \verb"cmr6", \verb"cmmi9", \verb"cmmi8", \verb"cmmi6",
\verb"cmsy9", \verb"cmsy8", \verb"cmsy6", \verb"cmbx9", \verb"cmbx8",
\verb"cmbx6", \verb"cmti9", \verb"cmti8", \verb"cmsl9", \verb"cmsl8",
\verb"cmcsc10", and \verb"cmcsc10 at 8 pt"), so you
will have to change these references to the appropriate \verb"am"
(almost modern) fonts
if that is all you have available
at your site.  The style file also assumes the
existence of the American Mathematical Society fonts
\verb"msxm9", \verb"msxm8", \verb"msxm6", \verb"msym9", \verb"msym8",
and \verb"msym6".  If you do
not have these fonts, then comment out the lines in amstexsiam.sty that
reference
them.

\heading{Control sequences defined in the amsppt style but not in the \AmSTeX\ SIAM style}
The commands
\verb"\affil", \verb"\endaffil", \verb"\address", \verb"\date", and
\verb"\thanks" have been redefined to give error messages when used, because
their function has been taken over by other constructions, mainly footnotes
to the title of the paper and the authors' names.
\bye
