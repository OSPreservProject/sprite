%------------------------------------------------------------------------------
% Beginning of dimacs-l.tex
%------------------------------------------------------------------------------
% This is a sample file for use with AMS-LaTeX. It provides an example of
% how to set up a file to be typeset with AMS-LaTeX.
%%%%%%%%%%%%%%%%%%%%%%%%%%%%%%%%%%%%%%%%%%%%%%%%%%%%%%%%%%%%%%%%%%%%
%
\documentstyle{dimacs-l}

\newtheorem{lemma}{Lemma}[section]
\newtheorem{theorem}{Theorem}[section]
\newtheorem{definition}{Definition}[section]

\theoremstyle{definition}
\newtheorem{example}{Example}[section]

\numberwithin{equation}{section}

\begin{document}

\title[MAXIMAL IDEALS IN SUBALGEBRAS OF $C(X)$]{Sample Paper for DIMACS,\\
On Maximal Ideals in Subalgebras of $C(X)$}
\author[AUTHOR ONE AND AUTHOR TWO]{Author One and Author Two}
\address{Department of Mathematics, Northeastern University, Boston,
Massachusetts 02115} % Research address for author one
\curraddr{Department of Mathematics and Statistics,
Case Western Reserve University, Cleveland, Ohio 43403}
\email{XYZ@@Math.AMS.com}
\address{Mathematical Research Section, School of Mathematical Sciences,
Australian National University, Canberra ACT 2601, Australia} %address for 
%                                                             author two

\subjclass{Primary 54C40, 14E20; Secondary 46E25, 20C20}
\date{July 2, 1991}

%  \thanks will become a 1st page footnote.
\thanks{The first author was supported in part by NSF
Grant \#000000.}
\thanks{The final version of this paper will be submitted for
publication elsewhere.}

\maketitle

\begin{abstract}
This paper is a sample prepared to illustrate for authors
the use of the \AmS-\LaTeX{} Version~1.1 package and the DIMACS documentstyle.

The file used to prepare this sample is {\bf DIMACS-L.tex}; an author
should use the coding in that file as a model.
\end{abstract}

\section{Introduction}

This sample paper illustrates the use of the AMSART style file of
\AmS-\LaTeX{} Version~1.1 with additional macros for the series {\it
DIMACS Series in Discrete Mathematics and Theoretical Computer Science}.
In this sample paper, brief instructions to authors will be interspersed
with mathematical text extracted from (purposely unidentified) published
papers.  For instructions on preparing mathematical text, the author is
referred to {\it The Joy of \TeX}, second edition, by Michael Spivak
\cite{spivak:jot} and {\it \LaTeX{}: A Document Preparation System} by
Leslie Lamport \cite{lamport:latex}.

\subsection{Top matter}
The input format and content of the top matter can be best understood
by examining the first part of the sample file {\bf DIMACS-L.tex}, up
through the \verb+\begin{document}+ instruction.

The top matter includes both elements that must be input by the author and a
few that are provided automatically.  The author names and the title that are
to appear in the running heads should be input between square brackets as an
option to the \verb+\author+ and \verb+\title+ commands, respectively. The full
names and title should be used unless they require too much space; in that
event, abbreviated forms should be substituted. In the top matter, the title is
input in caps and lowercase and will be set that way.  The author names should
be input in caps and lowercase; they will automatically be set in all caps.

For each author an address should be input.  If the current address is
different than the address where the research was carried out then both
addresses are given with the current address second and coded as indicated
in this sample file.   Following these addresses, an
address for electronic mail should be given, if one exists. Note that no
abbreviations are used in addresses, and complete addresses for each author
should be entered in the order that names appear on the title page.  Addresses
are considered part of the top matter but are set at the end of the paper,
following the references.

Subject classifications (\verb+\subjclass+) and acknowledgments
(\verb+\thanks+) are part of the top matter and will appear as footnotes at the
bottom of the first page.  Subject classifications
(\verb+\subjclass+) are required.  Use the 1991 Mathematics Subject Classification
that appears in annual indexes of {\it Mathematical Reviews\/}
beginning in 1990.  (The two-digit code from the Contents is not sufficient.)

Use \verb+\thanks+ for the footnotes that appear on the first page.  It is
generally desirable not to attach footnote numbers or symbols to titles or
author names used as headings.  If a footnote applies to only one author then
include this information in the footnote.

Papers published in proceedings of conferences are often abstracts or
preliminary versions.  In such a case, include a separate \verb+\thanks+
command stating ``The final
[detailed] version of this paper will be [has been] submitted for
publication elsewhere.''  Papers that are to be considered for review
by {\it Mathematical Reviews\/} should use instead
the following statement:
``This paper is in final form and no version of it will be submitted
for publication elsewhere.''

\subsection{Fonts}
The fonts used in this paper are from the Computer Modern family; they
should be available to all authors preparing papers with these macros.
However, the final copy may be set by the AMS using other fonts.  

\subsection{A mathematical extract}
The mathematical content of this sample paper has been extracted from
published papers, with no effort made to retain any mathematical sense.
It is intended only to illustrate the recommended manner of input.

Mathematical symbols in text should always be input in math mode as
illustrated in the following paragraph.

A function is invertible in $C(X)$ if it is never zero, and in $C^*(X)$ if
it is bounded away from zero. In an arbitrary $A(X)$, of course, there
is no such description of invertibility which is independent of the 
structure of the algebra. Thus in \S 2 we associate to each noninvertible
$f\in A(X)$ a $z$-filter $\cal Z (f)$ that is a measure of where
$f$ is ``locally'' invertible in $A(X)$. This correspondence extends to
one between maximal ideals of $A(X)$ and $z$-ultrafilters on $X$.
In \S 3 we use the filters $\cal Z (f)$ to describe the intersection of 
the free maximal ideals in any algebra $A(X)$. Finally, our main result
allows us to introduce the notion of $A(X)$-compactness of which 
compactness and realcompactness are special cases. In \S 4 we show how
the Banach-Stone theorem extends to $A(X)$-compact spaces.

\section{Theorems, lemmas, and other proclamations}

Theorems and lemmas are varieties of \verb+theorem+ environments.  In this
document, a \verb+theorem+ environment called \verb+lemma+ has been created,
which is used below. Also, there is a  proof, which is in the predefined
\verb+pf+ environment.  The lemma and proof below illustrate the use of 
the \verb+enumerate+ environment. 

\begin{lemma}
Let $f, g\in  A(X)$ and let $E$, $F$ be cozero
sets in $X$.
\begin{enumerate}
\item If $f$ is $E$-regular and $F\subseteq E$, then $f$ is $F$-regular.

\item If $f$ is $E$-regular and $F$-regular, then $f$ is $E\cup F$-%
regular.

\item If $f(x)\ge c>0$ for all $x\in E$, then $f$ is $E$-regular.

\end{enumerate}
\end{lemma}

\begin{pf}
\begin{enumerate}

\item  Obvious.

\item Let $h, k\in A(X)$ satisfy $hf|_E=1$ and $kf|_F=1$. Let
$w=h+k-fhk$. Then $fw|_{E\cup F}=1$.

\item Let $h=\max\{c,f\}$. Then $h|_E=f|_E$ and $h\ge c$. So $0<h^{-1}
\le c^{-1}$. Hence $h^{-1} \in C^*(X)\subseteq A(X)$, and 
$h^{-1} f|_E=1$. 

\end{enumerate}
\end{pf}

Another \verb+theorem+-type environment was defined at the beginning of this
document, called \verb+definition+. Here is an example of it:

\begin{definition}
For $f\in A(X)$, we define
\begin{equation}
\cal Z (f)=\{E\in Z[X]\: \text{$f$ is $E^c$-regular}\}.
\end{equation}
\end{definition}

\section{Roman type}

Numbers, punctuation, (parentheses), [brackets], $\{$braces$\}$, and
symbols used as tags should always be set in roman type.  The following
sample theorem illustrates how to code for roman type within the
statement of a theorem.

\begin{theorem}
Let $\cal G$ be a free nilpotent-of-class-$2$ group of rank
$\ge 2$ with carrier $G$ and let
$$m : G\times G \to Z$$
satisfy \rom{(2.21)}, \rom{(2.22)}, and \rom{(2.24)}, and define
$\kappa$ by \rom{(2.23)}.  Then this kappa-group is kappa-nilpotent
of class $2$ and kappa-metabelian, that is to say, it satisfies
\rom{S2} and \rom{S3}, but it is kappa-abelian if, and only if,
\begin{equation}
m(x,y) = -1\quad\text{for all $x, y \notin G'$}.
\end{equation}
\rom{(}Thus \rom{(3.1)} implies the trivial consequence
\rom{(2.1)}.\rom{)}  Assume now that \rom{(3.1)} does not hold,
so that the kappa-group is kappa-nonabelian.  Assume further that $m$
is not constant outside $G'$ \rom{(}inside $G'$ the values of $m$
clearly do not matter\rom{)}.  Then $\kappa$ is neither left nor right
linear, that is to say, neither \rom{S4} nor \rom{S5} holds:
\rom{I1} again holds, but none of \rom{I2--I5}.  As before,
\rom{I6} is equivalent to \rom{(2.25)}.  Now \rom{I7$'$}, however,
is equivalent to a condition similar to \rom{(2.25)}, namely
\begin{equation}
m(xz\sigma, yz\sigma) = m(x,y)\,.
\end{equation}
\end{theorem}

Letters used as abbreviations rather than as variables or constants
are set in roman type.  Use the control sequences \cite[p.~95]{spivak:jot}
for common mathematical functions and operators like $\log$ and $\lim$,
and use \verb+\cite+ when citing a reference.  The reference tag
will be {\bf bold} automatically, but you will need to set any
additional information in roman type as illustrated by the coding of
the previous sentence.

\section{References}

To produce a bibliography, use the environment named
\verb+thebibliography+. Input each reference as you would normally do in
\LaTeX{} \cite{lamport:latex}. Arrange the references in alphabetical
order by the last name of the first named author.  The references at the
end of this sample file have been chosen to illustrate the coding of the
most common types of references. Use the abbreviations of names of
journals as given in annual indexes of {\it Mathematical Reviews}.
The sample references have been labeled with numbers, using
\verb+\bibitem{...}+. To get letter labels use, for example,
\verb+\bibitem[C1]{...}+.

References are set with hanging indentation.  The widest label should be
entered as the argument of the {\tt thebibliography} environment, if you
are not using Bib\TeX{} (which automatically determines the widest
label). For example, a bibliography containing more than a hundred
references would require three-digit number labels:
\par\begin{centering}\smallskip
\begin{verbatim}
\begin{thebibliography}{000}
\end{verbatim}
\smallskip\end{centering}

\section{Figures}

Figures to be inserted later should be handled using \LaTeX's \verb+figure+
environment.  The amount of space left should equal the exact height of the
figure.  Extra space around the figure will be provided automatically.  The
positioning of figures may need to be changed to obtain the best possible page
layout.  Thus it is important to label your figures and use the labels in the
text when referring to figures.  The figure caption should be positioned below
the figure.

In most cases, figures will be rendered for consistency of style within a
book.  Please provide figure manuscript drawn in black ink with clean,
unbroken lines on nonabsorbent paper.

\begin{example}
For the link in Figure~\ref{firstfig}, the Massey product $\langle u_1,
u_2, u_3, u_4, u_5\rangle$ in $S^3-L$ is defined and consists of all
integer multiples of $\gamma_{1,5}$.  For the link in
Figure~\ref{firstfig}, the Massey product $\langle u_1, u_2, u_3, u_4,
u_5\rangle$ in $S^3-L$ contains the single element $\gamma_{1,5}$. 
Since the links in Figures~\ref{firstfig} and~\ref{otherfig} are
homotopic, the example indicates that Massey products in $S^3-L$ with
distinct $u_j$'s do not, in general, determine homotopy invariants of
the link.  For the link in Figure~\ref{firstfig} and the link in
Figure~\ref{otherfig}, the Massey product $\langle u_1, u_2, \dots,
u_5\rangle$ in $\{S^3-L_i\}_{i=1}^5$ contains the single element
$\gamma_{1,5}$.
\end{example}

%  art work measures 11.5pc for figure 1, 7pc for figure 2

\begin{figure}[t]
\vskip 11.5pc
\caption{Only the word {\it figure} is set cap/small cap.  Any other words are
regular text.\label{firstfig}}
\end{figure}

\begin{figure}[t]
\vskip 7pc
\caption{\label{otherfig}}
\end{figure}


\section{Other headings}

\subsection{A subsection}  We conclude by noting that another characterization
of $A$-compactness follows from Mandelker. We call a family $\cal S$ of closed
sets in $X\ A$-stable if every $f\in A(X)$ is bounded on some member of $\cal
S$. Then one can show that a space is $A$-compact if and only if  every
$A$-stable family of closed sets with the finite intersection property has
nonempty intersection.

\subsubsection{A second-level subheading}

This paragraph is included only to illustrate the appearance of a
sub-subsection.

%%%%%%%%%%%%%%%%%%%%%%%%%%%%%%%%%%%%%%%%%%%%%%%%%%%%%%%%%

\bibliographystyle{amsplain}

\begin{thebibliography}{10}

\bibitem{arnold:sing}
 V. L. Arnol$'$d, A. N. Varchenko, and S. M. Gusein-Zade,
 {\em Singularities of differentiable maps.} I,
  ``Nauka'', Moscow, 1982 (Russian);
English transl. Birkh\"auser, 1985.

\bibitem{arnold:sing2}
\bysame,
 {\em Singularities of differentiable maps.}~II,
  ``Nauka'', Moscow, 1984;
English transl., Birkh\"auser, 1988.


\bibitem{bass:jacobian}
H. Bass, E. H. Connell, and D. Wright,
{\em The Jacobian conjecture}, Bull. Amer. Math. Soc.
 {\bf 7} (1982), 287--330.

\bibitem{bass:flows}
H. Bass and G. H. Meisters,
{\em Polynomial flows in the plane}, Adv. in Math.
 {\bf 55} (1985), 173--203.

\bibitem{coomes:injectivity}
B. Coomes,
{\em Polynomial flows, symmetry groups, and conditions sufficient for 
        injectivity of maps},
 Ph.D. Thesis, Univ. Nebraska-Lincoln, 1988.


\bibitem{coomes:lorenz}
\bysame,
{\em The Lorenz system does not have a polynomial flow},
{J. Differential Equations} (to appear).

\bibitem{formanek:gener}
E. Formanek,
{\em Generating the ring of matrix invariants},
Lecture Notes in Math., vol. 1197,
Springer-Verlag, Berlin and New York,
1986, pp. 73--82.

\bibitem{meisters:poly}
\bysame,
{\em Polynomial flows on $\hbox{\bf R}^n$},
Proc. Semester on Dynamical Systems (Warsaw, Autumn 1986),
Springer-Verlag,
Berlin, Heidelberg, and New York
(to appear).

\bibitem{osher:shock}
S. Osher,
{\em Shock capturing algorithms for equations of mixed type},
 Numerical Methods for Partial Differential Equations
(S. I. Hariharan and T. H. Moulton, eds.),
  Longman, New York, 1986, pp. 305--322.

\bibitem{ostro:nonlin}
L. A. Ostrovsky,
{\em Nonlinear internal waves in a rotating ocean},
Part 2, Oceanology {\bf 18} (1978), 181--191.

\bibitem{petrov:ellip}
G. S. Petrov,
{\em Elliptic integrals and their nonoscillatory behavior},
Funktsional. Anal. i Pri\-lo\-zhen. {\bf 20} (1986), 46--49;
English transl. in Functional Anal. Appl. {\bf 20} (1986).

\bibitem{spivak:jot}
M. D. Spivak,
{\em The Joy of \TeX{}}, second edition, Amer. Math. Soc., Providence, R.~I., 
1990.

\bibitem{lamport:latex} Leslie Lamport, {\em \LaTeX{} -- A Document Preparation
System}, Addison-Wesley, Reading, Mass., 1986.

\end{thebibliography}

\end{document}

%------------------------------------------------------------------------------
% End of dimacs-l.tex
%------------------------------------------------------------------------------
