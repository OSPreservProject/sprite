\documentstyle[12pt]{laletter}  % type size of 12 points

%\documentstyle[11pt]{laletter} % type size of 11 points
%\documentstyle{laletter}       % type size of 10 points
%
% This is a sample file to show you typical input for printing a
% letter at the Los Alamos National Laboratory using LaTeX.
% 
% Notice that the '%' is the comment character.  The '%' and the
% rest of the line are ignored by LaTeX.  All of the LaTeX letter
% commands will appear in this file.  The ones not used by this
% file will be commented out.  To look at the body of the letter,
% skip down to the \begin{document} command.

% When you print out the DVI file that is generated from this
% source file, you will notice that the header doesn't look like
% pre-printed letter paper.  That is because \headerfonts{texfonts}
% is used here.  
% 
% If you are printing on a PostScript printer, use
% \headerfonts{postscript} to get a good looking header.  The
% \headerfonts{lafonts} command will also produce a nice header,
% but you will need to have installed the LANL header fonts.  If
% you use \headerfont{letterpaper}, put pre-printed letter paper
% in your printer or copying machine.
%
  \headerfonts{texfonts}
%  \headerfonts{postscript}
%  \headerfonts{lafonts}
%  \headerfonts{letterpaper}

  \bodyfonts{texfonts}
%  \bodyfonts{postscript}
  \typeface{tt}
%  \typeface{rm}

% With no date command, today's date will be printed.
%
  \date{July 19, 1989}

  \symbol{C-2}
%  \serialnumber{5-213}
  \mailstop{B253}
  \telephone{(505) 665-0859\\(FTS) 843-0859}
  \to{LaTeX Users \\
      Los Alamos National Laboratory \\
      Los Alamos, New Mexico 87545}
  \salutation {Dear LaTeX users:}
  \subject{typical letter}
%  \reference{Office Procedures Manual}
%  \nocallouts
%  \thru{Alicia J. Lujan, Division Leader\\Life Physics Division}
%  \via{Alicia J. Lujan, Division Leader\\Life Physics Division}

  \complimentaryclose {Sincerely yours,}
%  \complimentaryclose {Very truly yours,}
  \signature {Steve Sydoriak}
  \originator{SS}
  \signer {SS}
  \typist {SS}
  \encas
  \cy {CRMO (2), MS A111\\
       File}

%  \attachments{Graph, Gravitational Pull vs. Image Time, TP-3, MS B888}
%  \attachmentas
%  \attachmentsas
%  \distribution{K. C. Jordan, C-5, MS B777\\
%               T. S. Solomon, TP-1, MS B233}
%  \enc{Letter, Smith to Jones, June 25, 1986\\
%       Letter, Landau to Gresham, March 1, 1987}
%  \attachmentspagebreak
%  \cypagebreak
%  \distributionpagebreak
%  \encpagebreak

%  \shortletterstyle % For letters of ten lines or less.
%  \useattnasheader
%  \usesubjectasheader
%  \usetoasheader
%  \useotherasheader{Library Management Meeting}

%  \classlabel{u}  %  UNCLASSIFIED
%  \classlabel{c}  %  CONFIDENTIAL
%  \classlabel{s}  %  SECRET

%  \expandtopmargin{.5in}
%  \expandwidth{-.25in}

%
% If you use three or more \to commands, the addresses will be
% split into two columns.  Use \leftto and \rightto if you want to
% override this.
%
%  \leftto{Genie Electronics\\
%     \attn{Mr. Robert Mercer, Sales Manager}\\
%      P.O. Box 8501\\
%      Midwest City, OK 73110}
%  \leftto{John Binnington, Manager\\Technical Information Division\\
%    Brookhaven National Laboratory\\Associated Universities, Inc.\\
%    Upton, Long Island, NY  11973}
%  \rightto{Juanita L. Garcia\\Library Science Specialist\\Technical
%    Library\\Sandia National Laboratories\\Albuquerque, NM  87115}

%  \letterpaperhcorr{4pt}
%  \letterpapervcorr{-6pt}

%  \makemaillabels
%  \returnaddress{Ben Boyd \\ Los Alamos National Laboratory \\
%                 Group AB-11, MS D444 \\
%                 Los Alamos, NM 87545}
%  \skiplabels{4}

% The following default values set things up for producing
% labels in two columns of seven 1-1/2" X 4" labels each,
% suitable for reproducing onto Avery brand number 5362 address
% labels.

%  \maillabelheight{1.5in}
%  \maillabeltopmargin{.25in}

%  For 1" labels, use
%  \maillabelheight{1in}
%  \maillabeltopmargin{.5in}

%  For 2" labels,use
%  \maillabelheight{2in}
%  \maillabeltopmargin{.5in}

\begin{document}
\opening

This example shows what a typical letter might look like.  It uses
the texfonts option for the header and the body of the letter.
The texfonts option calls for use of TeX's Computer Modern fonts.
The body has the typewriter typeface with a point size of 12.

Remember that LaTeX interprets a blank line as the start of a new
paragraph, and that any of the special characters \#,\$,\%, \&,
\{, and \} must be preceded by a backslash.  To produce double
quotes in typewriter typeface, use the " key on your keyboard.  To
produce double quotes in Roman typeface, use `` and '' instead. 

The file named lettest.tex that was used to print this letter can
be used as a template to write your own letters.  All of the
preamble commands that can be used by LaTeX letters are shown in
lettest.tex.  Many of the commands are commented out.  They can be
activated by removing the \% at the beginning of the line.

The spacing and indentation of the preamble commands make the file
easier to read; the outcome of your file is not affected.  The
preamble begins with the $\backslash$documentstyle command and
continues to the $\backslash$begin\{document\} command where the
document section starts.  Your file must have a
$\backslash$end\{document\} command to indicate the end of the
letter.

The texfonts option was chosen for the header of this letter
because the Computer Modern fonts are available on any
installation of TeX\@.  Use the postscript, lafonts, or
letterpaper options in the $\backslash$headerfonts command to
obtain a good looking header.

There is a draft version of the {\em LaTeX Letter Reference} on CFS
in any subdirectory of /latexletter.  It is an stexted, ASCII file
that can be viewed on any screen or printed on any printer.
The final version of the {\em LaTeX Letter Reference} will be
available from the Computer Information Center in September or
October.

See {\em LaTeX, A Document Preparation System} by Leslie
Lamport to learn all about the LaTeX commands that you can put
in the body of your letter.

\closing
\end{document}
